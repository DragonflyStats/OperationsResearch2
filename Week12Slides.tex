5 Consider the IP:
max x1 + 2x2
−x1 + x2 ≤ 10
15x1 + 16x2 ≤ 240
x1, x2 ≥ 0 and integer.
%==========================================================================%
You will be asked to partly solve the IP, using the Branch and Bound Method.
You should draw an enumeration tree/diagram to keep track of your
progress — draw the enumeration tree on an otherwise blank page.
%==========================================================================%
The corresponding Simplex Tableau (transforming the max problem into a
min problem) is:
T0 =
0 −1 −2 0 0
10 −1 1 1 0
240 15 16 0 1

%==========================================================================%

\begin{frame}
\begin{itemize}
\item In contrast to the material used in Week 11 , the indicator row is listed at the top of the tableau
\item The $b$ column is listed on the left hand side of the tableau
\end{itemize}
\end{fame}
%==========================================================================%
N.B. The Simplex Method and the Dual Simplex Method are stated on
the last page of this paper.
You will partly solve the IP, following the steps on the next pages — remember
to draw and fill in an enumeration tree.
%==========================================================================%
(a) Apply one iteration of the Simplex Method to the starting tableau (T0)
and show that the Simplex Tableau now takes the form: 3%
20 −3 0 2 0
10 −1 1 1 0
80 31 0 −16 1
%==========================================================================%
(b) After a second iteration of the Simplex Method the Simplex Tableau
now takes the form: (N.B. Do not perform the arithmetic!)
T2 =
27.74 0 0 0.45 0.10
12.58 0 1 0.48 0.03
2.58 1 0 −0.52 0.03
%==========================================================================%
(i) Explain why this Tableau is LP optimal. 1%
(ii) What is the solution to the LP Relaxation of the IP? 1%
(iii) Why is this not IP-optimal? 1%
(iv) What bound can you conclude for the optimal value of the objective
for the original IP? 1%
%==========================================================================%

(v) What are the two possible branches on x1 and what are the two
possible branches on x2? 1%
(c) Consider the right branch corresponding to a lower bound on x2 (x2 ≥
?).
%==========================================================================%
(i) The LP-optimal tableau T2 must be amended by adding an extra
row & column corresponding to the extra constraint. Explain
carefully why the new tableau T3 (with 4 rows and 6 columns)
takes the form: 2%
T3 =
27.74 0 0 0.45 0.10 0
12.58 0 1 0.48 0.03 0
2.58 1 0 −0.52 0.03 0
−13 0 −1 0 0 1
(ii) Which element of T3 should you pivot on next? Explain carefully. 1%
%==========================================================================%
(iii) The resulting tableau is
T4 =
27.74 0 0 0.45 0.10 0
12.58 0 1 0.48 0.03 0
2.58 1 0 −0.52 0.03 0
−0.42 0 0 0.48 0.03 1
(iv) Explain carefully why T4 is infeasible and update your enumeration
tree. 1%
(d) Consider the left branch corresponding to an upper bound on x2 (x2 ≤
?).
(i) The LP-optimal tableau T2 must be amended by adding an extra
row & column corresponding to the extra constraint. Explain
carefully why the new tableau T5 (with 4 rows and 6 columns)
takes the form: 2%
T5 =
27.74 0 0 0.45 0.10 0
12.58 0 1 0.48 0.03 0
2.58 1 0 −0.52 0.03 0
12 0 1 0 0 1
%==========================================================================%
(ii) Which element of T5 should you pivot on next? Explain carefully. 1%
(iii) The resulting tableau is
T6 =
27.74 0 0 0.45 0.10 0
12.58 0 1 0.48 0.03 0
2.58 1 0 −0.52 0.03 0
−0.58 0 0 −0.48 −0.03 1
Using the Dual Simplex Method, which element should you pivot
on next so that the tableau is LP-optimal? Explain carefully. 2%
%==========================================================================%

(iv) The resulting tableau is:
T7 =
27.20 0 0 0 0.07 0.93
12 0 1 0 0 1
3.20 1 0 0 0.07 −1.07
1.20 0 0 1 0.07 −2.07
(v) Is T7 LP-optimal? Explain carefully. 1%
(vi) Is T7 IP-optimal? Explain carefully. 1%
(e) Consider the left branch corresponding to an upper bound on x1 (x1 ≤
?).
%=============================================================================%

(i) As T7 is not IP-optimal it must be amended by adding an extra row
& column corresponding to the extra constraint. Explain carefully
why the new tableau T8 (with 5 rows and 7 columns) takes the
form: 2%
T8 =
27.20 0 0 0 0.07 0.93 0
12 0 1 0 0 1 0
3.20 1 0 0 0.07 −1.07 0
1.20 0 0 1 0.07 −2.07 0
3 1 0 0 0 0 1
%==========================================================================%
(ii) Which element of T8 should you pivot on next? Explain carefully. 1%
%==========================================================================%
(iii) The resulting tableau is
T9 =
27.20 0 0 0 0.07 0.93 0
12 0 1 0 0 1 0
3.20 1 0 0 0.07 −1.07 0
1.20 0 0 1 0.07 −2.07 0
−0.20 0 0 0 −0.07 1.07 1
%==========================================================================%
(iv) Using the Dual Simplex Method, which element of T9 should you
pivot on next? Explain carefully. 1%

(v) The resulting tableau is:
27 0 0 0 0 2 1
12 0 1 0 0 1 0
3 1 0 0 0 0 1
1 0 0 1 0 −1 1
3 0 0 0 1 −16 −15
%==========================================================================%
(vi) Does this tableau represent the optimal solution to the IP? Explain
carefully why or why not. 1%
(vii) What are the solution x and the objective value z associated with
the tableau? 1%
