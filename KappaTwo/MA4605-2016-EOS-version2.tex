

\documentclass[a4paper,12pt]{article}
%%%%%%%%%%%%%%%%%%%%%%%%%%%%%%%%%%%%%%%%%%%%%%%%%%%%%%%%%%%%%%%%%%%%%%%%%%%%%%%%%%%%%%%%%%%%%%%%%%%%%%%%%%%%%%%%%%%%%%%%%%%%%%%%%%%%%%%%%%%%%%%%%%%%%%%%%%%%%%%%%%%%%%%%%%%%%%%%%%%%%%%%%%%%%%%%%%%%%%%%%%%%%%%%%%%%%%%%%%%%%%%%%%%%%%%%%%%%%%%%%%%%%%%%%%%%
\usepackage{eurosym}
\usepackage{vmargin}
\usepackage{amsmath}
\usepackage{graphics}
\usepackage{epsfig}
\usepackage{subfigure}
\usepackage{fancyhdr}
%\usepackage{listings}
\usepackage{framed}
\usepackage{graphicx}
\usepackage{amsmath}
\usepackage{chngpage}
%\usepackage{bigints}


\setcounter{MaxMatrixCols}{10}
%TCIDATA{OutputFilter=LATEX.DLL}
%TCIDATA{Version=5.00.0.2570}
%TCIDATA{<META NAME="SaveForMode" CONTENT="1">}
%TCIDATA{LastRevised=Wednesday, February 23, 2011 13:24:34}
%TCIDATA{<META NAME="GraphicsSave" CONTENT="32">}
%TCIDATA{Language=American English}

%\pagestyle{fancy}
%\setmarginsrb{20mm}{0mm}{20mm}{25mm}{12mm}{11mm}{0mm}{11mm}
%\lhead{MA4413} \rhead{Mr. Kevin O'Brien}
%\chead{Statistics For Computing}
%\input{tcilatex}

\begin{document}
\begin{center}
\includegraphics[scale=0.65]{images/shieldtransparent2}
\end{center}

\begin{center}
\vspace{1cm}
\large \bf {FACULTY OF SCIENCE AND ENGINEERING} \\[0.5cm]
\normalsize DEPARTMENT OF MATHEMATICS AND STATISTICS \\[1.25cm]
\large \bf {REPEAT EXAMINATION PAPER 2016} \\[1.5cm]
\end{center}

\begin{tabular}{ll}
MODULE CODE: MA4605 & SEMESTER: Autumn 2016 \\[1cm]
MODULE TITLE: Chemometrics & DURATION OF EXAM: 2.5 hours \\[1cm]
LECTURER: Mr. Kevin O'Brien & GRADING SCHEME: 100 marks \\
& \phantom{GRADING SCHEME:} \footnotesize {60\% of module grade} \\[0.8cm]
EXTERNAL EXAMINER: Prof. J. King & \\
\end{tabular}
\bigskip
\begin{center}
{\bf INSTRUCTIONS TO CANDIDATES}
\end{center}

{\noindent \\ Scientific calculators approved by the University of Limerick can be used. \\
Formula sheet and statistical tables areprovided at the end of the exam paper.\\
Students must attempt any 4 questions from 5.}
\newpage



%=================================================================================%

The gold content of a concentrated sea-water sample was determined by using
atomic-absorption spectrometry with the method of standard additions.

The results obtained were as follows:

Gold Addition   & Absorbance
(ng/ml)

Gold <-c(30,40,50,60,70,0,10,20,80,70)
Abso <-c(0.415,0.472,0.528,0.579,0.641,0.271,0.323,0.369,0.678,0.752)

% latex table generated in R 3.3.1 by xtable 1.8-2 package
% Sat Oct 29 21:05:55 2016
\begin{table}[ht]
	\centering
	\begin{tabular}{rrr}
		\hline
		& Gold & Abso \\ 
		\hline
		1 & 30.00 & 0.41 \\ 
		2 & 40.00 & 0.47 \\ 
		3 & 50.00 & 0.53 \\ 
		4 & 60.00 & 0.58 \\ 
		5 & 70.00 & 0.64 \\ 
		6 & 0.00 & 0.27 \\ 
		7 & 10.00 & 0.32 \\ 
		8 & 20.00 & 0.37 \\ 
		9 & 80.00 & 0.68 \\ 
		10 & 70.00 & 0.75 \\ 
		\hline
	\end{tabular}
\end{table}


