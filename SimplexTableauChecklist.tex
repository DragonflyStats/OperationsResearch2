\documentclass{beamer}

\usepackage{framed}

\begin{document}
%======================================%
\begin{frame}
\vspace{-0.5cm}
	\noindent \textbf{Checklist}
	
	\begin{itemize}
        \item \textbf{Important:} Read Each Question Carefully - dont rush into doing things you are not asked to do
		\item Picking Pivot Column (most negative value in indicator row  a.k.a. the top row)
		\item Picking Pivot Row 
		
		\begin{itemize}
		
		\item[$\ast$] \textit{(focus only on \textbf{positive} values entries in Pivot column only.)} 
		\item[$\ast$]	Divide the ``Barrier" column by pivot column for each corresponding
			entry - computing ratios
			\item[$\ast$] Choose the smallest (positive) ratio (i.e. closed to zero)
		\end{itemize}
		\item Picking Pivot Point (Intersection of Pivot Row and Pivot Column)
	\end{itemize}
	
\end{frame}
%==============================================================%
\begin{frame}
	\vspace{-0.5cm}
	\noindent \textbf{Checklist - Continued}
	
	\begin{itemize}
		\item Performing Elementary Row Operations 
			\begin{itemize}
				\item[$\ast$] Turn Pivot Point into a value of 1
				\item[$\ast$] Make other values in Pivot Column 0
			\end{itemize}
		\item \textbf{Important:} Recognize when iteration process is complete 
		\item \textbf{Important:} Recognize when iteration process isn't complete - particularly important when new constraints are added.

	\end{itemize}
	
\end{frame}
%==============================================================%
\begin{frame}
\vspace{-0.5cm}
	\noindent \textbf{Checklist - Continued}
	
	\begin{itemize}
				\item Recognize when optimal feasible solution has been found
				\item Recognize infeasibility 
		\item State the solution of Tableau(i.e. for LP relaxation)
		\item (Recognize which variables necessarily have a value of zero).
		\item LP Relaxation.
		\item Recognize that LP optimality does not equate to IP optimality.
		\item Adding bounds (based on LP optimal solution) . Be able to state what these new constraints are.
	\end{itemize}
\end{frame}
%==============================================================%
\begin{frame}
	\noindent \textbf{Addition of Constraints to Simplex Tableau}
	
	\begin{itemize}
\item Important: Construction of New Constraints further to branch and bound.
\item This will involve adding new rows and columns to the tableau.
\item Remark: Exam 2012 Q1 Part D is very useful to practice with in this regard.
\item See slides for \textbf{exceedance constraints} (i.e. $x_i \geq  k$)
\end{itemize}
\end{frame}
%==============================================================%
\begin{frame}
\noindent \textbf{Dual Simplex Method}	\\
\noindent \textbf{(In General  : Tranpose/Reverse of Simplex Method)}
\begin{itemize}
	\item Pick most negative value from LHS column (i.e. the barrier column)
	\item We pick the associated row
	\item In the example below - we'd pick the last row, the row for -3.
\end{itemize}
{
	\large
\begin{center}
\begin{tabular}{|c||c|c|c|c|c|}
	\hline 25 & $\ldots$ & $\ldots$ & $\ldots$ & $\ldots$ & $\ldots$ \\  
\hline	\hline 15 & $\ldots$ & $\ldots$ & $\ldots$ & $\ldots$ & $\ldots$ \\ 
	\hline 4 & $\ldots$ & $\ldots$ & $\ldots$ & $\ldots$ & $\ldots$ \\  
	\hline \textbf{-3} & $\ldots$ & $\ldots$ & $\ldots$ & $\ldots$ & $\ldots$ \\ 
	\hline 
\end{tabular} 	
\end{center}
}
\end{frame}
%================================================= %
\begin{frame}
\noindent \textbf{Dual Simplex Method}	\\
	\begin{itemize}
		\item We are mainly interested in negative values on this row. Concentrate only on them.
	\end{itemize}
{
	\large
	\begin{center}
		\begin{tabular}{|c||c|c|c|c|c|}
			\hline 25 & $\ldots$ & $\ldots$ & $\ldots$ & $\ldots$ & $\ldots$ \\  \hline
			\hline 15 & $\ldots$ & $\ldots$ & $\ldots$ & $\ldots$ & $\ldots$ \\ 
			\hline 4 & $\ldots$ & $\ldots$ & $\ldots$ & $\ldots$ & $\ldots$ \\  
			\hline \textbf{-3} & $\ldots$ & -5 & $\ldots$ & -2 & $\ldots$ \\ 
			\hline 
		\end{tabular} 	
	\end{center}
}
\end{frame}
\begin{frame}
\noindent \textbf{Dual Simplex Method}	\\
	\begin{itemize}
		\item Compare this values to the values in the top row. Compute the ratios. 
	\end{itemize}
	{
		\large
		\begin{center}
			\begin{tabular}{|c||c|c|c|c|c|}
				\hline 25 & $\ldots$ & 10 & $\ldots$ & 8 & $\ldots$ \\  \hline
				\hline 15 & $\ldots$ & $\ldots$ & $\ldots$ & $\ldots$ & $\ldots$ \\ 
				\hline 4 & $\ldots$ & $\ldots$ & $\ldots$ & $\ldots$ & $\ldots$ \\  
				\hline \textbf{-3} & $\ldots$ & -5 & $\ldots$ & -2 & $\ldots$ \\ 
				\hline 
			\end{tabular} 	
		\end{center}
	}
\begin{itemize}
	\item[$\ast$] Ratio: 10 /-5  = -2
	\item[$\ast$] Ratio: 8 /-2 = -4
\end{itemize}
\bigskip
\begin{itemize}
	\item Dont Expect any positive values here. 
	\item Choose highest value (i.e closest to 0)
\end{itemize}
\end{frame}
\end{document}