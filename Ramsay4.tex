2.4 Value Iteration
Value iteration is based on approximating an infinite horizon
problem by a finite horizon problem.
Let vn(i) be the optimal value starting in state i with n stages to
go and v(i) be the value of starting in state i in the infinite stage
problem.
Let cmax < ∞ be the maximum immediate cost (reward) incurred
in a stage and β be the discount factor. It is reasonably simple to
show that
lim
n→∞
vn(i)=v(i)
|v(i) − vn(i)|<
β
n
cmax
1 − β
.
1 \end{frame}  \begin{frame} \frametitle{Integer Programming}     
Idea of value iteration
Also, for large n, it is intuitively clear that the optimal action to
take in state i is optimal or close to being optimal in the infinite
horizon problem.
It follows that the optimal stationary policy and values can be
approximated by the optimal policy and values in the first step of a
problem with horizon n.
Returning to Example 2.2.1, it can be seen that the optimal policy
when 2 or more stages are left is the same as the optimal policy in
the infinite horizon problem, i.e. fix when in state B and leave
when in state G.
2 \end{frame}  \begin{frame} \frametitle{Integer Programming}     
Example 2.3.3
Suppose an individual has 2 possible actions A, B and there are 3
states 1, 2, 3 (poor, intermediate, good). The transition
probabilities and payoffs are
Payoff 3 2 1
A, 3 10 0.2 0.4 0.4
A, 2 5 0.1 0.3 0.6
A, 1 2 0.0 0.1 0.9
B, 3 5 0.9 0.1 0.0
B, 2 3 0.6 0.3 0.1
B, 1 1 0.4 0.4 0.2
3 \end{frame}  \begin{frame} \frametitle{Integer Programming}     
Recursive solution of the infinite horizon problem
With 1 stage to go an individual should just maximise their
immediate reward, i.e. take action A in any state. We have
R1(3) = 10, R1(2) = 5, R1(1) = 2.
The maximum error in estimating R(j) (the value in the infinite
horizon problem) at this stage is 10β
1−β = 40.
4 \end{frame}  \begin{frame} \frametitle{Integer Programming}     
Recursive solution of the infinite horizon problem-ctd.
With two stages to go in state 3
R2(3)=max{10 + 0.8[0.2R1(3) + 0.4R1(2) + 0.4R1(1)],
5 + 0.8[0.9R1(3) + 0.1R1(2)]}
=max{13.84, 12.6} = 13.84.
The optimal action is A.
With two stages to go in state 2
R2(2)=max{5 + 0.8[0.1R1(3) + 0.3R1(2) + 0.6R1(1)],
3 + 0.8[0.6R1(3) + 0.3R1(2) + 0.1R1(1)]}
=max{7.96, 9.16} = 9.16.
The optimal action is B.
5 \end{frame}  \begin{frame} \frametitle{Integer Programming}     
Recursive solution of the infinite horizon problem-ctd.
With two stages to go in state 1
R2(1)=max{2 + 0.8[0.1R1(2) + 0.9R1(1)],
1 + 0.8[0.2R1(3) + 0.4R1(2) + 0.4R1(1)]}
=max{3.84, 4.84} = 4.84.
The optimal action is B.
Calculating in a similar way, we derive the following optimal values
and actions, an(i). These calculations can be done using a simple
programme or in Excel.
6 \end{frame}  \begin{frame} \frametitle{Integer Programming}     
Recursive solution of the infinite horizon problem-ctd.
n Rn(3) an(3) Rn(2) an(2) Rn(1) an(1)
1 10 A 5 A 2 A
2 13.84 A 9.16 B 4.84 B
3 16.69 A 12.23 B 7.69 B
4 19.05 A 14.56 B 10.05 B
5 20.92 A 16.44 B 11.92 B
6 22.42 A 17.94 B 13.42 B
7 23.63 A 19.14 B 14.63 B
8 24.59 A 20.10 B 15.59 B
9 25.35 A 20.87 B 16.35 B
10 25.97 A 21.49 B 16.97 B
It can be seen that for n ≥ 2 the optimal policy is to take action A
when in state 3, otherwise take action B. This is in fact the
optimal policy in the infinite horizon problem.
7 \end{frame}  \begin{frame} \frametitle{Integer Programming}     
Advantages and disadvantages of these 2 methods
The calculations involved in value iteration are simpler than the
calculations used in policy iteration. However, value iteration does
have some problems.
The main problem is convergence. The condition that
an+1(i) = an(i), ∀i ∈ 1, 2, . . . , m (i.e. the optimal set of actions
with n + 1 steps to go is the same as the optimal set of actions
with n steps to go) is not sufficient to state that the optimal policy
in the infinite horizon problem is {an(i)}
m
j=1.
8 \end{frame}  \begin{frame} \frametitle{Integer Programming}     
Convergence of value iteration
Convergence can be checked at stage n by calculating
n(i) = |vn(i) − wn(i)| for each state i, where wn(i) is the cost
incurred by taking the second best action with n steps to go and
thereafter using the optimal policy.
For example, if
δn = max
1≤i≤m
{n(i)} ≥ cmaxβ
n
1 − β
,
then we may stop and state that {a(i)}
m
j=1 is the optimal policy in
the infinite stage problem.
It should be noted that this is a rather conservative condition and
convergence may be very slow, especially when β is close to 1. In
the example given above, the convergence criterion is met after 18
iterations.
9 \end{frame}  \begin{frame} \frametitle{Integer Programming}     
Advantages of policy iteration and hybrid algorithms
The calculations involved in policy iteration are more complex.
However, if πn+1 = πn, then πn is the optimal policy in the infinite
horizon problem.
For these reasons in complex problems where the set of stationary
strategies is large, a hybrid algorithm may be used.
A few iterations of the value iteration algorithm allow us to find a
good solution cheaply. Afterwards, the policy iteration algorithm
can be implemented using this good solution as the starting point.
10 \end{frame}  \begin{frame} \frametitle{Integer Programming}     
Policy iteration and random time horizons
It should also be noted that the policy iteration algorithm can also
be used in stochastic problems where the process finishes when an
absorbing state is attained. In such problems the time horizon is
random (see Tutorial Sheet 2).
11 \end{frame}  \begin{frame} \frametitle{Integer Programming}     
2.3.2 Average cost problems
Deterministic problems As before, we consider only stationary
strategies and a finite state space.
Let a(i) be the action carried out in state i.
Since the state space is finite, the state must at some point return
to a state that has previously been visited. A deterministic process
will then cycle (we must carry out the same sequence of actions in
all the following cycles).
The states observed in such a cycle are referred to as recurrent
states. The number of recurrent states must be equal to the cycle
length.
It should be noted that the average costs incurred in such a cycle
do not depend on the initial state (possibly non-recurrent).
12 \end{frame}  \begin{frame} \frametitle{Integer Programming}     
Deterministic problems
It should be noted that the assumption of a finite state space may
be relaxed to the assumption that the set of recurrent states under
a stationary strategy is finite (see following example).
Suppose the cycle length is Lπ and the cost per cycle is Ccyc,π.
The average costs incurred are
Cav,π =
Ccyc,π
Lπ
.
The optimal policy π minimises Cav,π.
13 \end{frame}  \begin{frame} \frametitle{Integer Programming}     
Example 1.4.4 - The equipment replacement problem
Consider the problem of replacing a car (see Tutorial Sheet 1).
I want to minimise the average costs of running a car. I start with
a new car.
The annual costs of running a car which will be x years old at the
end of the year are 1000 + 2000x.
At the end of each year I can make one of two decisions: a) sell my
present car and buy a new car. The cost of a new car is 20 000
and the price I get for my old car is 20000
x+1 , or b) continue.
Calculate the optimal average costs.
14 \end{frame}  \begin{frame} \frametitle{Integer Programming}     
Solution to Example 2.3.4
First note that it cannot be optimal to never sell the car, since the
annual running costs tend to infinity (i.e. the average running
costs per year tend to infinity).
The average annual cost of buying a new car every year is 13 000
(buy for 20 000, running costs 3 000 and sale price 10 000), i.e.
finite.
It follows that it must be optimal to sell the car at some given age.
Let k (finite) be the youngest age at which it is optimal to sell the
car. It is clear that the age of the car at each decision point (at
the end of a year) in a cycle goes from 1 to k. Hence, the set of
recurrent states is finite.
15 \end{frame}  \begin{frame} \frametitle{Integer Programming}     
Solution to Example 2.3.4
Let Cav,k be the average costs incurred under such a policy.
Since the cycle length is k (after k years we have a new car again),
the costs incurred in a cycle Ccyc,k are:
a) Running costs:
X
k
i=1
1000 + 2000k = 1000k + 1000k(k + 1) = 1000k
2 + 2000k
b) Costs of selling the old car and buying a new one:
20000 −
20000
k+1 .
16 \end{frame}  \begin{frame} \frametitle{Integer Programming}     
Solution to Example 2.3.4-ctd.
Hence, we have
Cav,k=
Ccyc,k
k
=
1000k
2 + 2000k + 20000 − 20000/(k + 1)
k
=1000k + 2000 + 20000
k
−
20000
k(k + 1)
=1000k + 2000 + 20000
k + 1
.
17 \end{frame}  \begin{frame} \frametitle{Integer Programming}     
Solution to Example 2.3.4-ctd.
We could find the optimal solution by evaluating the average cost
for k = 1, 2, . . . until the average costs start to increase and then
pick the k that minimises the average costs.
There are two problems with this approach
1. The optimal value of k may be large (i.e. a large
number of evaluations need to be carried out).
2. There may be multiple local minima of the optimal
cost function. Hence, this procedure is not
guaranteed to find the global minimum.
18 \end{frame}  \begin{frame} \frametitle{Integer Programming}     
Solution to Example 2.3.4-ctd.
We can find the solution using differentiation, but remembering
that k is defined to be an integer.
If the average cost function has a unique (global) minimum at k,
then bkc and dke are the only candidates for the global minimiser
of the average cost function over the set of integers.
bkc is the integer part of k (the largest integer not larger than k)
and dke is the ceiling of k (the smallest integer not smaller than k).
19 \end{frame}  \begin{frame} \frametitle{Integer Programming}     
Solution to Example 2.3.4-ctd.
If there is a local minimum of the average cost function at k,
then bkc and dke are candidates for the global minimiser of the
average cost function over the set of integers.
In this case we can find the global minimiser by evaluating the
average costs incurred under all the candidate strategies.
20 \end{frame}  \begin{frame} \frametitle{Integer Programming}     
Solution to Example 2.3.4-ctd.
We have
dCav,k
dk = 1000 −
20000
(k + 1)2
This derivative is 0 when k = 3.47, negative for k < 3.47 and
positive for k > 3.47. Hence, the cost function has a global
minimum at k = 3.47.
Since k is defined to be an integer, using the argument above the
optimal value of k must be either 3 or 4.
21 \end{frame}  \begin{frame} \frametitle{Integer Programming}     
Solution to Example 2.3.4-ctd.
By direct calculation, we have
Cav,3 = 10000; Cav,4 = 10000
It can be seen that selling each car at age 3 or 4 years is optimal.
The average cost incurred under either of these stationary policies
is 10000 Euro/year.
Note that there are also infinitely many non-stationary strategies
which are optimal. These strategies involve sometimes having a car
for 3 years and sometimes for 4 years.
22 \end{frame}  \begin{frame} \frametitle{Integer Programming}     
Stochastic Problems
It will be assumed that the number of states is finite and each
state is recurrent under any policy.
Given this, the expected time go from the present state to any
other (including returning to the present state) is finite.
Let Lπ(i, j) be the expected time required to go from state i to
state j and let Cπ(i, j) be the expected costs incurred in going
from state i to state j under the strategy π.
23 \end{frame}  \begin{frame} \frametitle{Integer Programming}     
Average cost incurred under a policy
The average cost incurred under a policy π, Cav,π is given by
Cav,π =
Cπ(i, i)
Lπ(i, i)
, i ∈ {1, 2, . . . , m}.
It should be noted that this average cost is independent of i (it
does not depend on which state the process is in at the beginning
of a ”cycle”).
We now consider how to calculate Cπ(i, i) and Lπ(i, i).
24 \end{frame}  \begin{frame} \frametitle{Integer Programming}     
Expected length of a cycle starting in state i
Under a given policy, there is an m × m matrix defining the
transition probabilities pi,j,π, where pi,j,π is the probability of going
from state i to state j in one stage under policy π.
Supppose we wish to calculate Lπ(i, j), the expected number of
stages required to go from state i to state j under the policy π.
Suppose at stage 0, we are in state i. We will be in state j at stage
1 with probability pi,j,π. In this case, the number of stages required
to get to state j is 1.
25 \end{frame}  \begin{frame} \frametitle{Integer Programming}     
Expected length of a cycle starting in state i-ctd.
At stage 1 with probability pi,k,π, we will be in state k. In this case
the expected number of additional stages required to get to j is
Lπ(k, j) [i.e. in total on average 1 + Lπ(k, j) stages will be
required].
Applying the law of total probability to expected values, we have
Lπ(i, j)=pi,j,π +
X
1≤k≤m;k6=j
pi,k,π(1 + Lπ(k, j))
=1 + X
1≤k≤m;k6=j
pi,k,πLπ(k, j)
By varying i from 1 to m and fixing the finishing state j, we obtain
a system of m linear equations that can be used to derive
Lπ(1, j), Lπ(2, j), . . . , Lπ(m, j).
26 \end{frame}  \begin{frame} \frametitle{Integer Programming}     
Expected costs incurred in a cycle starting in state i-ctd.
Now consider Cπ(i, j). Let ci,k,π be the immediate cost of
transition from state i to state k in one stage (i.e. the cost of the
action plus the cost of being in the appropriate state - can be
defined to be the initial state or the final state as long as this
choice is made consistently).
Applying the law of total probability as above, we have
Cπ(i, j) = pi,j,πci,j,π +
X
1≤k≤m;k6=j
pi,k,π(ci,k,π + Cπ(k, j))
Again, by varying i from 1 to m, we obtain a system of m linear
equations that can be used to derive
Cπ(1, j), Cπ(2, j), . . . , Cπ(m, j).
27 \end{frame}  \begin{frame} \frametitle{Integer Programming}     
Example 2.3.5
Calculate the average reward obtained in the infinite horizon,
non-discounted problem described in Example 2.3.3 under the
policy π, where π is the strategy: always take the action B
28 \end{frame}  \begin{frame} \frametitle{Integer Programming}     
Solution to Example 2.3.5
The first thing to do is to define the transition matrix under π.
In state 1 we take action B and hence the following state is 1 with
probability 0.2, 2 with probability 0.4 and 3 with probability 0.4.
Hence, p1,1,π = 0.2, p1,2,π = 0.4, p1,3,π = 0.4. Arguing similarly,
the transition matrix is given by
1 2 3
1 0.2 0.4 0.4
2 0.1 0.3 0.6
3 0.0 0.1 0.9
29 \end{frame}  \begin{frame} \frametitle{Integer Programming}     
Solution to Example 2.3.5-ctd.
We now calculate the expected transition times to state 3, say. We
have
Lπ(i, 3) = 1 +X
2
k=1
pi,k,πLπ(k, 3).
Hence,
Lπ(1, 3)=1 + p1,1,πLπ(1, 3) + p1,2,πLπ(2, 3)
=1 + 0.2Lπ(1, 3) + 0.4Lπ(2, 3)
Lπ(2, 3)=1 + p2,1,πLπ(1, 3) + p2,2,πLπ(2, 3)
=1 + 0.1Lπ(1, 3) + 0.3Lπ(2, 3)
Lπ(3, 3)=1 + p3,1,πLπ(1, 3) + p3,2,πLπ(2, 3)
=1 + 0.1Lπ(2, 3).
30 \end{frame}  \begin{frame} \frametitle{Integer Programming}     
Solution to Example 2.3.5-ctd.
We can use the first and second equations to find Lπ(1, 3) and
Lπ(2, 3). Rearranging the first equation and substituting it into the
second,
Lπ(1, 3)=5 + 2Lπ(2, 3)
4
Lπ(2, 3)=1 + 0.1
5 + 2Lπ(2, 3)
4
+ 0.3Lπ(2, 3).
Solving these equations, we obtain
Lπ(2, 3) = 45
26
; Lπ(1, 3) = 55
26
; Lπ(3, 3) = 61
52
31 \end{frame}  \begin{frame} \frametitle{Integer Programming}     
Solution to Example 2.3.5-ctd.
We now calculate the expected rewards obtained from going from
state i to state 3 under π, Rπ(i, 3). We have
Rπ(i, j) = pi,j,πri,j,π +
X
1≤k≤m;k6=j
pi,k,π(ri,k,π + Rπ(k, j)),
where ri,k,π is the reward of transition from i to k in one step (this
is just the reward gained in the initial state). Since r1,k,π = 1,
r2,k,π = 3, r3,k,π = 5, we have
Rπ(1, 3)=p1,3,π + p1,1,π(1 + Rπ(1, 3)) + p1,2,π(1 + Rπ(2, 3))
=1 + 0.2Rπ(1, 3) + 0.4Rπ(2, 3)
Rπ(2, 3)=3p2,3,π + p2,1,π(3 + Rπ(1, 3)) + p2,2,π(3 + Rπ(2, 3))
=3 + 0.1Rπ(1, 3) + 0.3Rπ(2, 3)
Rπ(3, 3)=5p3,3,π + p3,1,π(5 + Rπ(1, 3)) + p3,2,π(5 + Rπ(2, 3))
=5 + 0.1Rπ(2, 3)
32 \end{frame}  \begin{frame} \frametitle{Integer Programming}     
Solution to Example 2.3.5-ctd.
This system of equations may be solved in the same way as the
system above to give
Rπ(1, 3) = 95
26
; Rπ(2, 3) = 125
26
; Rπ(3, 3) = 285
52
.
The average reward obtained under policy π is thus
Rav,π =
Rπ(3, 3)
Lπ(3, 3) =
285
61
≈ 4.6721.
33 \end{frame}  \begin{frame} \frametitle{Integer Programming}     
Policy Iteration
Suppose we have calculated Lπ(i, i) and Cπ(i, i).
We can check whether the policy π could be improved by changing
the action used when in state i.
Consider the strategy (a(i), π), which uses the action a(i) in state
i and otherwise uses the action defined by π.
34 \end{frame}  \begin{frame} \frametitle{Integer Programming}     
Perturbed strategies in discounted value problems and
average value problems
Note that the interpretation of such a perturbed strategy is slightly
different to the notion of a perturbed strategy we used in
discounted problems.
In discounted problems, a perturbed strategy uses a different
action at stage 1, but then always follows the original strategy.
In average value problems, a perturbed strategy changes the action
every time one given state is entered.
35 \end{frame}  \begin{frame} \frametitle{Integer Programming}     
Average costs under modified strategy: Policy
improvement
The expected time required to return to state i under this policy is
L(a(i),π)
(i, i) = 1 + X
1≤k≤m;k6=i
pi,k,a(i)Lπ(k, i).
Note that this follows from the fact that if the state does not
immediately return to i, then the policy π is followed until the end
of the cycle.
This expression can simply be evaluated using previous calculations.
Similarly, the expected cost incurred in a cycle starting in state i
under this policy is
C(a(i),π)
(i, i) = pi,i,a(i)ci,i,a(i)+
X
1≤k≤m;k6=i
pi,k,a(i)
(ci,k,a(i)+Cπ(k, i)).
36 \end{frame}  \begin{frame} \frametitle{Integer Programming}     
Average costs under modified strategy: Policy
improvement
We have
Cav,(a(i),π) =
C(a(i),π)
(i, i)
L(a(i),π)
(i, i)
.
If Cav,(a(i),π) < Cav,π, then we can improve π by changing the
action taken in state i to a(i).
37 \end{frame}  \begin{frame} \frametitle{Integer Programming}     
Policy iteration algorithm
At each step of the policy iteration algorithm, we use this method
to see if we can improve the action taken in each of the states
1, 2, . . . , m.
If the action cannot be improved in any of the states, then we have
found the optimal policy.
Otherwise, we replace the strategy π with the strategy π
0 made up
of the set of the best actions in each state found using the
algorithm above and repeat the policy improvement step.
38 \end{frame}  \begin{frame} \frametitle{Integer Programming}     
Example 2.3.6
Using the policy iteration algorithm, find a policy that is better
than π: always use B.
39 \end{frame}  \begin{frame} \frametitle{Integer Programming}     
Solution to Example 1.4.6
Since we have calculated {Lπ(i, 3)}i=1,2,3, we consider changing
the action carried out in state 3, i.e. the perturbed policy
(a(3), π), where a(3) = A. This is the policy use A in state 3,
otherwise use B.
The reward gained at stage 1 is 10 and the transition probabilities
are p3,1,A = 0.4, p3,2,A = 0.4, p3,3,A = 0.2.
40 \end{frame}  \begin{frame} \frametitle{Integer Programming}     
Solution to Example 2.3.6-ctd.
The expected time to return to state 3 is given by
L(A,π)
(3, 3)=1 + p3,1,ALπ(1, 3) + p3,2,ALπ(2, 3)
=1 + 0.4 ×
55
26
+ 0.4 ×
45
26
=
66
26
.
The expected reward obtained in such a cycle is given by
R(A,π)
(3, 3)=10 + p3,1,ARπ(1, 3) + p3,2,ARπ(2, 3)
=10 + 0.2 ×
95
26
+ 0.4 ×
125
26
=
348
26
.
The expected average reward is given by
Rav,(A,π) =
R(A,π)
(3, 3)
L(A,π)
(3, 3) =
348
66
≈ 5.2727.
Since Rav,(A,π) > Rav,π, the policy defined by (A, π) i.e. play A in
state 3, otherwise play B is better than the strategy of always
playing B.
41 \end{frame}  \begin{frame} \frametitle{Integer Programming}     
2.4 Optimal stopping problems - The classical marriage
problem
It is assumed that a person (searcher) observes a sequence of
objects (candidates to be a marriage partner).
The value of the i-th candidate seen is Xi
, where the Xi are
non-negative, independent and identically distributed (i.i.d.)
random variables with known density function f (x).
The cost of observing a candidate is c. On observing a candidate
the searcher must decide to either accept the candidate, thus
ceasing to search and obtaining a reward of Xi − ci, or reject the
candidate and continue. It is assumed that c < E(Xi).
Once a candidate has been rejected, it cannot be later accepted.
42 \end{frame}  \begin{frame} \frametitle{Integer Programming}     
2.4.1 The form of the optimal policy for the finite horizon
problem
Let the time horizon be n.
It is clear that starting at stage i, the expected total reward from
search can be maximised by maximising the expected reward from
future search (the search costs incurred before time i do not affect
the rule that should be used from time i onwards).
Suppose the optimal expected reward from search starting at stage
i (ignoring previous costs) is vi
.
43 \end{frame}  \begin{frame} \frametitle{Integer Programming}     
The Bellman equation for the finite horizon problem
By accepting a candidate of value Xi
, the searcher obtains a
reward of Xi − c. By rejecting this candidate, the searcher obtains
an expected reward of vi+1 − c.
Hence, an individual should accept a candidate at stage i, if and
only if its value is greater than vi+1 (i.e. the expected reward from
future search).
The Bellman equation is given by
vi=E[max{Xi
, vi+1}] − c =
Z ∞
0
max{x, vi+1}f (x)dx − c
=
Z vi+1
0
vi+1f (x)dx +
Z ∞
vi+1
xf (x)dx − c
=vi+1P(Xi < vi+1) + Z ∞
vi+1
xf (x)dx − c
44 \end{frame}  \begin{frame} \frametitle{Integer Programming}     
The Bellman equation for the finite horizon problem
If the time horizon is n, then vn = E(Xn) − c, since the last
candidate should always be chosen when none has been previously
chosen.
The optimal reward from search is given by v1.
45 \end{frame}  \begin{frame} \frametitle{Integer Programming}     
Alternative form of the Bellman equation
Since
E[Xi
|Xi > vi+1] =
R ∞
vi+1
xf (x)dx
P(Xi > vi+1)
,
the above equation can be written in the form
vi = vi+1P(Xi < vi+1) + E[Xi
|Xi > vi+1]P(Xi > vi+1) − c
46 \end{frame}  \begin{frame} \frametitle{Integer Programming}     
Example 2.4.1
Suppose the objects have i.i.d. values from the uniform
distribution on [0, 1] and the cost of observing an object is 0.1.
i. Derive a recursive formula for the optimal expected
reward in the finite horizon problem.
ii. Calculate the optimal expected reward when a
maximum of 5 objects can be observed.
47 \end{frame}  \begin{frame} \frametitle{Integer Programming}     
Solution to Example 2.4.1-i
i. We have f (x) = 1, x ∈ [0, 1].
vi=E[max{Xi
, vi+1}] − c
=
Z vi+1
0
vi+1f (x)dx +
Z ∞
vi+1
xf (x)dx − c
=vi+1P(Xi < vi+1) + Z ∞
vi+1
xf (x)dx − c
=v
2
i+1 +
Z 1
vi+1
xdx − 0.1 = v
2
i+1 +
1 − v
2
i+1
2
− 0.1
=
1 + v
2
i+1
2
− 0.1.
48 \end{frame}  \begin{frame} \frametitle{Integer Programming}     
Solution to Example 2.4.1-ii
ii. Working recursively v5 = E(X5) − 0.1 = 0.4.
v4=
1 + v
2
5
2
− 0.1 = 0.48
v3=
1 + v
2
4
2
− 0.1 = 0.5152
v2=
1 + v
2
3
2
− 0.1 = 0.5327
v1=
1 + v
2
2
2
− 0.1 = 0.5419.
49 \end{frame}  \begin{frame} \frametitle{Integer Programming}     
The optimal policy
The threshold used at stage i is vi+1 (i.e. accept a candidate if
his/her value is greater than the expected reward from future
search).
Thus in the problem considered above
1. At the last stage (stage 5) accept any candidate.
2. At stage 4 accept candidates of value ≥ 0.4.
3. At stage 3 accept candidates of value ≥ 0.48.
4. At stage 2 accept candidates of value ≥ 0.5152.
5. At stage 1 accept candidates of value ≥ 0.5327.
50 \end{frame}  \begin{frame} \frametitle{Integer Programming}     
Discounted payoffs
The rewards may be discounted over time according to the number
of candidates seen. When the discount factor is β, the Bellman
equation is given by
vi=βE[max{Xi
, vi+1}] − c
=β
 Z vi+1
0
vi+1f (x)dx +
Z ∞
vi+1
xf (x)dx!
− c
=β

vi+1P(Xi < vi+1) + Z ∞
vi+1
xf (x)dx!
− c.
At stage i, it is optimal to accept a candidate if Xi ≥ vi+1.
51 \end{frame}  \begin{frame} \frametitle{Integer Programming}     
2.4.2 The form of the optimal policy for infinite horizon
problems
Let v be the expected reward in the infinite horizon problem. If the
first candidate is rejected, then the expected future reward
(ignoring the cost of observing the first observation) is again
simply v. Hence, the Bellman equation is given by
v=E[max{Xi
, v}] − c
=
Z v
0
vf (x)dx +
Z ∞
v
xf (x)dx − c
=vP(Xi < v) + Z ∞
v
xf (x)dx − c.
This gives us an equation for v. The optimal policy is to accept a
candidate iff its value is ≥ v (i.e. a fixed threshold rule).
52 \end{frame}  \begin{frame} \frametitle{Integer Programming}     
Example 2.4.2
Find the optimal threshold rule for the infinite-horizon version of
the problem in Example 2.4.1.
53 \end{frame}  \begin{frame} \frametitle{Integer Programming}     
Solution to Example 2.4.2
The Bellman equation is
v=vP(Xi < v) + Z ∞
v
xf (x)dx − c
=v
2 +
Z 1
v
xdx − 0.1 = 1 + v
2
2
− 0.1
54 \end{frame}  \begin{frame} \frametitle{Integer Programming}     
Solution to Example 2.4.2
This leads to the quadratic equation v
2 − 2v + 0.8 = 0.
There are two roots of this equation: v1 =
2+√
0.8
2
and
v2 =
2−
√
0.8
2
.
The first solution cannot be the correct solution, since the reward
from search is bounded above by 1. Hence, we have
v =
2 −
√
0.8
2
≈ 0.5528.
This threshold is equal to the expected reward from search.
55 \end{frame}  \begin{frame} \frametitle{Integer Programming}     
