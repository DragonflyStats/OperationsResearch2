%%-  http://www.inf.ufpr.br/aurora/disciplinas/topicosia2/livros/search/integer.pdf

\section{Facility Location}

We consider a facihty location problem. A chemical company owns four
factories that manufacture a certain chemical in raw form. The company would
like to get in the business of refining the chemical. It is interested in building
refining facilities, and it has identified three possible sites. 

\end{frame}
%==============================================================%

\begin{frame}


Table 3.1 contains
variable costs, fixed costs, and weekly capacities for the three possible refining
facility sites, and weekly production amounts for each factory. The variable
costs are in dollars per week and include transportation costs. The fixed costs
are in dollars per year. The production amounts and capacities are in tons per
week.
\end{frame}
%==============================================================%

\begin{frame}
\item The decision maker who faces this problem must answer two very different
types of questions: questions that require numerical answers (for example,
how many tons of chemical should factory / send to the site-y refining facility
each week?) and questions that require yes-no answers (for example, should
the site-y facility be constructed?). 
\item While we can easily model the first type
of question by using continuous decision variables (by letting Xij equal the number of tons of chemical sent from factory / to site j each week), we cannot
do this with the second. We need to use integer variables. If we let yj equal 1
if the site-y refining facihty is constructed and 0 if it is not, we quickly arrive
at an IP formulation of the problem: 
\end{frame}
%================================================================= %

The objective is to minimize the yearly cost, the sum of the variable costs
(which are measured in dollars per week) and the fixed costs (which are measured
in dollars per year). The first set of constraints ensures that each factory's
weekly chemical production is sent somewhere for refining. Since factory 1
produces 1000 tons of chemical per week, factory 1 must ship a total of 1000
tons of chemical to the various refining facilities each week. The second set
of constraints guarantees two things: (1) if a facihty is open, it will operate
at or below its capacity, and (2) if a facility is not open, it will not operate
at all. If the site-1 facility is open (yi = 1) then the factories can send it up
to 1500^1 = 1500 • 1 = 1500 tons of chemical per week. If it is not open 

(_yj =: 0), then the factories can send it up to 1500};i = 1500 0 = 0 tons per
week.
This introductory example demonstrates the need for integer variables. It
also shows that with integer variables, one can model simple logical requirements
(if a facility is open, it can refine up to a certain amount of chemical;
if not, it cannot do any refining at all). It turns out that with integer variables,
one can model a whole host of logical requirements. One can also model fixed
costs, sequencing and scheduling requirements, and many other problem aspects.

