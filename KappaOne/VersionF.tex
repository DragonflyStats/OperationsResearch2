\documentclass[a4paper,12pt]{article}
%%%%%%%%%%%%%%%%%%%%%%%%%%%%%%%%%%%%%%%%%%%%%%%%%%%%%%%%%%%%%%%%%%%%%%%%%%%%%%%%%%%%%%%%%%%%%%%%%%%%%%%%%%%%%%%%%%%%%%%%%%%%%%%%%%%%%%%%%%%%%%%%%%%%%%%%%%%%%%%%%%%%%%%%%%%%%%%%%%%%%%%%%%%%%%%%%%%%%%%%%%%%%%%%%%%%%%%%%%%%%%%%%%%%%%%%%%%%%%%%%%%%%%%%%%%%
\usepackage{eurosym}
\usepackage{vmargin}
\usepackage{amsmath}
\usepackage{graphics}
\usepackage{framed}
\usepackage{epsfig}
\usepackage{subfigure}
\usepackage{fancyhdr}

\setcounter{MaxMatrixCols}{10}
%TCIDATA{OutputFilter=LATEX.DLL}
%TCIDATA{Version=5.00.0.2570}
%TCIDATA{<META NAME="SaveForMode"CONTENT="1">}
%TCIDATA{LastRevised=Wednesday, February 23, 201113:24:34}
%TCIDATA{<META NAME="GraphicsSave" CONTENT="32">}
%TCIDATA{Language=American English}

\pagestyle{fancy}
\setmarginsrb{20mm}{0mm}{20mm}{25mm}{12mm}{11mm}{0mm}{11mm}
\lhead{MA4603} \rhead{Kevin O'Brien} \chead{Midterm
Assessment Paper - Version A } %\input{tcilatex}

\begin{document}
\begin{center}
	\includegraphics[scale=0.60]{images/shieldtransparent2}
\end{center}

\begin{center}
	\vspace{1cm}
	\large \bf {FACULTY OF SCIENCE AND ENGINEERING} \\[0.5cm]
	\normalsize DEPARTMENT OF MATHEMATICS AND STATISTICS \\[1.25cm]
	\large \bf {MID-TERM ASSESSMENT EXAMINATION 1} \\[1.5cm]
\end{center}

%\begin{tabular}{ll}
%	MODULE CODE: MA4603 & SEMESTER: Autumn 2016\\[1cm]
%	MODULE TITLE: Science Mathematics 3  & DURATION OF EXAM: 45 minutes \\[1cm]
%	LECTURER: Mr. Kevin O'Brien & GRADING SCHEME: 20 marks \\
%	& \phantom{GRadiC} \footnotesize {20\% of total module marks} \\[0.2cm]
%	\\[1cm]
%\end{tabular}
\begin{center}
	{\bf INSTRUCTIONS TO CANDIDATES}
\end{center}
\begin{itemize}
	\item This exam will start at 12:05, and will last 45 minutes.
	
	\item Each question will be worth either 5 Marks. There are 20 Marks worth of questions.
	\item All questions must be attempted (LENS students please see below)
	
	\item Write all of your answers in the exam script. Write the script number on any other documents you submit.
	
	\item It is your responsibility to return the script to collection box. An audit of scripts will take place immediately after the exam. If your script is account for in that audit,  you are deemed to be absent, and will receive no marks.
	
	\item \textbf{IMPORTANT for LENS Student:}
	Specifically approved LENS students have to answer any 3 of the 4 questions .
	\begin{itemize}
		\item Question 1 Part A and Question 1 Part B count as a single question. If you choose to attempt Question 1, you must answer both parts.
	\end{itemize}
\end{itemize}	
	


\newpage
\section*{Attempt ALL questions}
\subsection*{Q1. Descriptive Statistics (4 Marks)}


\subsection*{Q2. Dixon Q Test For Outliers (5 Marks)}

The typing speeds for one group of 12 Engineering students were recorded both at the beginning of year 1 of their studies. The results (in words per minute) are given below:

\begin{center}
	\begin{tabular}{|c|c|c|c|c|c|}
		\hline
		% Subject& A& B& C& D& E &F &G &H \\ \hline
		118 & 146 & 149 & 142 & 170& 153\\ \hline
		137 & 161 & 156& 165&  178& 159
		\\ \hline
	\end{tabular}
\end{center}
Use the Dixon Q-test to determine if the lowest value (118) is an outlier. You may assume a significance level of 5\%.
\begin{itemize}
	\item[i.](1 Mark)	State the Null and Alternative Hypothesis for this test.
	\item[ii.](1 Marks) Compute the test statistic
	\item[iii.](1 Mark) State the appropriate critical value.
	\item[iv.](1 Mark) What is your conclusion to this procedure
\end{itemize}



\subsubsection*{Part B} %NORMAL %8 MARKS
Assume that the length of injected moulded plastic components are normally distributed with a mean of 12.5mm and a standard deviation of 2.5mm.  Calculate the corresponding probability for the following measurements occurring on an individual component.

\begin{itemize}
	\item [i.](2 Marks)	Between 12.5 and 15mms,
	\item [ii.](2 Marks) Less than 10 mms,
	\item [iii.](2 Marks) Between 12 and 15 mms,
	\item [iv.](2 Marks) Less than 10.3 mms.
\end{itemize}
\noindent Use the normal tables to determine the probabilities for the above exercises. You are required to show all of your workings.
\newpage
(Write Your Answers Here)
\newpage
\vspace{0.25cm}
\subsubsection*{Part B} %10 MARKS
Data on the durations (measured in months) were collected for a random sample of product development projects.
The durations for these development projects were collected and tabulated as follows:

\begin{table}[ht]
	\begin{center}
		\begin{tabular}{|rrrrrrrr|}
			
			\hline
			12 & 11 & 20 & 19 & 18 & 9 & 16 & 15 \\
			\hline
		\end{tabular}
	\end{center}
\end{table}

\begin{itemize}
	\item[i.](1 Mark) Calculate the mean of the project durations.
	\item[ii.](2 Marks) Calculate the variance for this sample.
	\item[iii.](1 Mark) Calculate the standard deviation for this sample.
\end{itemize}



\subsection*{Q3. Normal Distribution (5 Marks)} % Normal %6 MARKS
Assume that the diameter of a critical component is normally distributed with a Mean of 200mm and a Standard Deviation of 4mm. You are required  to estimate the approximate probability of the following measurements occurring on an individual component.
\begin{itemize}
	\item[i.](1 Mark)	Greater than 203.9mm
	\item[ii.](2 Marks) Less than 195.2 mm
	\item [iii.](2 Marks) Between 94.2 and 103 mm
\end{itemize}
\bigskip
\noindent Use the normal tables to determine the probabilities for the above exercises. You are required to show all of your workings.
\newpage
(Write Your Answers Here)
\newpage
\vspace{0.25cm}

\subsection*{Q4. Confidence Interval for a Proportion (5 Marks)}
The strength of dosage of a plant growth enhancement chemical is often measured by the proportion of plants that grow faster. A particular dosage of the chemical is fed to 119 plants of these plants, 97 actually show faster growth.

\begin{itemize}
	\item[i.] (1 Mark) Calculate a point estimate $\hat{p}$ for the proportion of plants that grow faster due to the dosage. 									 
	\item[ii.] (2 Marks)  What is the standard error of the estimate? 			
	\item[iii.] (2 Marks) Find a 95\% confidence interval for the proportion. 					
\end{itemize}
\newpage
\subsection*{Critical Values for Dixon Q Test}
{
	\Large
	\begin{center}
		\begin{tabular}{|c|c|c|c|}
			\hline  N  & $\alpha=0.10$  & $\alpha=0.05$  & $\alpha=0.01$  \\ \hline
			3  & 0.941 & 0.970 & 0.994 \\ \hline
			4  & 0.765 & 0.829 & 0.926 \\ \hline
			5  & 0.642 & 0.710  & 0.821 \\ \hline
			6  & 0.560 & 0.625 & 0.740 \\ \hline
			7  & 0.507 & 0.568 & 0.680  \\ \hline
			8  & 0.468 & 0.526 & 0.634 \\ \hline
			9  & 0.437 & 0.493 & 0.598 \\ \hline
			10 & 0.412 & 0.466 & 0.568 \\ \hline
			11 & 0.392 & 0.444 & 0.542 \\ \hline
			12 & 0.376 & 0.426 & 0.522 \\ \hline
			13 & 0.361 & 0.410 & 0.503 \\ \hline
			14 & 0.349 & 0.396 & 0.488 \\ \hline
			15 & 0.338 & 0.384 & 0.475 \\ \hline
			16 & 0.329 & 0.374 & 0.463 \\ \hline
		\end{tabular} 
	\end{center}
}

\end{document}