Exam13

Explain briefly why the following strategy for the solution of I.P.’s
is not useful: “Solve the L.P. relaxation then round off the components
of the solution to the nearest integers”.
\end{frame}
%================================================%
\begin{frame}

(b) Consider the Integer Linear Program (IP):
max 5x1 + 2x2
5x1 + x2 \leq 5/2
10x1 + 6x2 \leq 45
x1, x2 \leq 0 and integer.
\end{frame}
%================================================%
\begin{frame}

The corresponding Simplex Tableau (transforming the max problem
into a min problem) is:
0 -5 -2 0 0
5/2 5 1 1 0
45 10 6 0 1
N.B. The Simplex Method and the Dual Simplex Method are stated
on the last page of this paper.
\end{frame}
%================================================%
\begin{frame}

(i) Apply one iteration of the Simplex Method and show that the Simplex
Tableau now takes the form: 4
2.5 0 -1 1 0
0.5 1 0.2 0.2 0
40 0 4 -2 1

\end{frame}
%================================================%
\begin{frame}

(ii) After a second iteration of the Simplex Method the Simplex Tableau
now takes the form: (N.B.do not perform the arithmetic!)
5 5 0 2 0
2.5 5 1 1 0
30 -20 0 -6 1
Explain why this Tableau is optimal. 1
\end{frame}
%================================================%
\begin{frame}
(iii) Explain why the solution to the LP Relaxation of the IP is x1 =
0, x2 = 2.5 and why we must now branch on x2 and what are the
two branches? 1
\end{frame}
%================================================%
\begin{frame}
(iv) Consider the branch S0 : x2 \leq 2.
\begin{itemize{
\item[A.] First show that the basic variable x2 may be expressed in
terms of the non-basic variables x1 & x3 as:
x2 = 2.5 − 5x1 − x3. 1
\item[B.] Substitute this expression for x2 into the S0 branch constraint
and show that it takes the form −5x1 − x3 + s = −0.5. (The
variable s is the slack variable for the constraint x2 \leq 2.) 1
\end{itemize}

\end{frame}
%================================================%
\begin{frame}
\frametitle{2013 Exam)
\large

C. Show that the Simplex Tableau with the addition of this constraint
takes the form: 1
5 5 0 2 0 0
2.5 5 1 1 0 0
30 -20 0 -6 1 0
-0.5 -5 0 -1 0 1
\end{frame}
%================================================%
\begin{frame}
\frametitle{2013 Exam)
\large


(v) Explain why this tableau is optimal but infeasible. 1
(vi) Apply one iteration of the Dual Simplex Method (DSM) to this
tableau and show that the Simplex Tableau now takes the form: 4
4.5 0 0 1 0 1
2 0 1 0 0 1
32 0 0 -2 1 -4
0.1 1 0 0.20 0 -0.2
\end{frame}
%================================================%
\begin{frame}
\frametitle{2013 Exam)
\large
\begin{itemize}
\item[(vii)] Is this tableau LP optimal? Is it integer feasible? Explain. What
is the solution to this LP relaxation? 1
\item[(viii)] What branching should you now make?
\end{itemize}
\end{frame}
%================================================%
\begin{frame}
\frametitle{2013 Exam)
\large
(ix) Having chosen a suitable branching, explain why the new tableau
for the left-hand branch (x1 = 0) is:
4.5 0 0 1 0 1
2 0 1 0 0 1
32 0 0 -2 1 -4
0.1 1 0 0.2 0 -0.2
-0.1 0 0 -0.2 0 0.2

A. What row and column is selected for pivoting with DSM? 1
B. Check that, after pivoting with DSM, x1 and x2 remain basic
with x1 = 0 and x2 = 2. (No need to update the full tableau,
just the top left-hand element and the second-last element in
the the left-hand column.) and hence find the integer optimal
value of z? 
\end{frame}
%================================================%
\
