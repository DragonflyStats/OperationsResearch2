\documentclass[a4paper,12pt]{article}
%%%%%%%%%%%%%%%%%%%%%%%%%%%%%%%%%%%%%%%%%%%%%%%%%%%%%%%%%%%%%%%%%%%%%%%%%%%%%%%%%%%%%%%%%%%%%%%%%%%%%%%%%%%%%%%%%%%%%%%%%%%%%%%%%%%%%%%%%%%%%%%%%%%%%%%%%%%%%%%%%%%%%%%%%%%%%%%%%%%%%%%%%%%%%%%%%%%%%%%%%%%%%%%%%%%%%%%%%%%%%%%%%%%%%%%%%%%%%%%%%%%%%%%%%%%%
\usepackage{eurosym}
\usepackage{vmargin}
\usepackage{amsmath}
\usepackage{graphics}
\usepackage{framed}
\usepackage{epsfig}
\usepackage{subfigure}
\usepackage{fancyhdr}

\setcounter{MaxMatrixCols}{10}
%TCIDATA{OutputFilter=LATEX.DLL}
%TCIDATA{Version=5.00.0.2570}
%TCIDATA{<META NAME="SaveForMode"CONTENT="1">}
%TCIDATA{LastRevised=Wednesday, February 23, 201113:24:34}
%TCIDATA{<META NAME="GraphicsSave" CONTENT="32">}
%TCIDATA{Language=American English}

\pagestyle{fancy}
\setmarginsrb{20mm}{0mm}{20mm}{25mm}{12mm}{11mm}{0mm}{11mm}
\lhead{MA4603} \rhead{Kevin O'Brien} \chead{Midterm
Assessment Paper - Version A } %\input{tcilatex}

\begin{document}
\begin{center}
	\includegraphics[scale=0.60]{images/shieldtransparent2}
\end{center}

\begin{center}
	\vspace{1cm}
	\large \bf {FACULTY OF SCIENCE AND ENGINEERING} \\[0.5cm]
	\normalsize DEPARTMENT OF MATHEMATICS AND STATISTICS \\[1.25cm]
	\large \bf {MID-TERM ASSESSMENT EXAMINATION 1} \\[1.5cm]
\end{center}

\begin{tabular}{ll}
	MODULE CODE: MA4603 & SEMESTER: Autumn 2016\\[1cm]
	MODULE TITLE: Science Mathematics 3  & DURATION OF EXAM: 45 minutes \\[1cm]
	LECTURER: Mr. Kevin O'Brien & GRADING SCHEME: 20 marks \\
	& \phantom{GRadiC} \footnotesize {20\% of total module marks} \\[0.2cm]
	\\[1cm]
\end{tabular}
\begin{center}
	{\bf INSTRUCTIONS TO CANDIDATES}
\end{center}
\begin{itemize}
	\item Attempt all questions.
	\item Show your working clearly.
	\item Please write your name and student number on each page you submit.

\end{itemize}
\newpage
\section*{Attempt ALL questions}


\subsection*{Q1. Dixon Q Test For Outliers (4 Marks)}

The typing speeds for one group of 12 Engineering students were recorded both at the beginning of year 1 of their studies. The results (in words per minute) are given below:

\begin{center}
	\begin{tabular}{|c|c|c|c|c|c|c|}
		\hline
		% Subject& A& B& C& D& E &F &G &H \\ \hline
		121 & 146 & 150 &149 &142 &170& 153\\ \hline
		137 & 161 & 156& 165& 137& 178& 159
		\\ \hline
	\end{tabular}
\end{center}
Use the Dixon Q-test to determine if the lowest value (121) is an outlier. You may assume a significance level of 5\%.
%Calculate a 95\% confidence interval for the difference between the mean number of marks obtained by males and females in the population of school leavers as a whole.
%(7 marks)

\begin{itemize}
	\item[i.] (1 Mark) Formally state the null hypothesis and the alternative hypothesis.
	\item[ii.] (1 Mark) Compute the Test Statistic.
	\item[iii.] (2 Mark) By comparing the Test Statistic to the appropriate Critical Value, state your conclusion for this test.
\end{itemize}
\newpage
%\newpage
\subsection*{Q2. Normal Distribution (5 Marks)} % Normal %6 MARKS
Assume that the diameter of a critical component is normally distributed with a Mean of 100mm and a Standard Deviation of 5mm. You are required  to estimate the approximate probability of the following measurements occurring on an individual component.
\begin{itemize}
	\item [i.](1 Mark)	Greater than 103mm
	\item [ii.](2 Marks) Less than 94.2 mm
	\item [iii.](2 Marks)[$\ast$] Between 94.2 and 103 mm
\end{itemize}
\bigskip
\subsection*{Q2. Descriptive Statistics A (3 Marks)} % 5 Marks
Consider the following data set of seven numbers:

\begin{center}
	\textbf{\texttt{29 14 17 30 19 25 13}}
\end{center}
% 4 Marks

\noindent For this sample, compute the following descriptive statistics:
\begin{itemize}
	%\item[a.] (1 Mark) The median,
	\item[a.] (1 Mark) The mean,
	\item[b.] (1 Mark) The variance,
	\item[c.] (1 Mark) The standard deviation.
\end{itemize}


\end{document}