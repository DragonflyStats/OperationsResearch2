Operations Research 2
Lecturer: David Ramsey
Room B-2026
david.ramsey@ul.ie
www.ul.ie/ramsey
\end{frame}  
\begin{frame} 
\frametitle{Integer Programming}     
Recommended Text
Operations Research- An Introduction. Hamdy A. Taha
(003/TAH)
The 7th edition is the most up to date in the library. The library
has 4 copies of this edition. In addition, there are around 25 copies
of older editions. Also,
Introduction to Operations Research. Hillier and Lieberman.
Course notes and tutorial sheets will be available on the Internet.
2 
\end{frame}  
%=========================================================%
\begin{frame} 
\frametitle{Integer Programming}     
Module Outline
1. Integer Programming. Applications to the travelling
salesperson’s problem and the knapsack (packing) problem.
2. Dynamic Programming. Deterministic. Stochastic. Finite
horizon. Infinite horizon problems. Discounting and average-cost
problems.
3. Game Theory. Matrix Games. Minimax solutions. Nash
equilibria. Correlated Equilibrium. Applications to economics and
biology.
3 
\end{frame}  
\begin{frame} 
\frametitle{Integer Programming}     
CHAPTER 1 - INTEGER PROGRAMMING AND THE
BRANCH AND BOUND ALGORITHM
1.1 The definition of an integer programming problem
An integer programming problem can be defined by the description
of a linear programming problem together with the constraint that
the variables must be integers. It is assumed that the variables are
constrained to be non-negative.
% end of Slide 4 
\end{frame}  
%=====================================================================================%
\begin{frame} 
\frametitle{Integer Programming}     


Example of an integer programming problem
max z = 4x1 + 5x2
subject to
x1 + x2 ≤ 5
6x1 + 10x2 ≤ 45,
where x1, x2 are non-negative integers.
Note that we can describe any integer programming problem as a
maximisation problem (the minimisation of 4x1 + 5x2 is equivalent
to the maximisation of −4x1 − 5x2). The function z = 4x1 + 5x2 is
called the objective function.
5 
\end{frame}  
\begin{frame} 
\frametitle{Integer Programming}     
\noindent \textbf{1.2 The Branch and Bound Method - The initial bound}
\begin{itemize}
\item 
We may find an initial upper bound on the optimal value in the
integer programming problem by solving the corresponding linear
programming problem.
\item Consider the problem given above. If the values of x1 and x2 at the
optimal solution to this problem are integers, then this must be the
optimal solution to the integer programming problem.
\end{itemize}
% end of slide 6 
\end{frame}  
\begin{frame} 
\frametitle{Integer Programming}     
The Branch and Bound Method - The initial branching
step
Suppose x1 = c at this optimal solution, where c is not an integer.
We can split the initial linear programming problem into two linear
programming problems.
The first problem is obtained by adding the constraint x1 ≤ bcc
(bcc is the largest integer not greater than c).
The second is obtained by adding the constraint x1 ≥ dce (dce is
the smallest integer not less than c).
7 
\end{frame}  
\begin{frame} 
\frametitle{Integer Programming}     
\noindent \textbf{The initial branching step}
\begin{itemie}
\item The union of the feasible sets for these two LP problems is the
feasible set for the initial LP problem with the set of (x1, x2)
satisfying bcc < x1 < dce removed.
\item Since no feasible solutions in which both x1 and x2 are integers
have been removed (see diagram below, the dots show the feasible
points in the integer programming problem), the optimal solution
to the integer programming problem must lie in one of the feasible
regions of these two new problems.
8 
\end{frame}  
\begin{frame} 
\frametitle{Integer Programming}     
\noindent \textbf{Solution of the IP problem}\\
Consider the linear programming problem
max z = 4x1 + 5x2
subject to
x1 + x2≤5
6x1 + 10x2≤45,
The feasible region is illustrated in the figure on the next slide.
9 
\end{frame}  
\begin{frame} 
\frametitle{Integer Programming}     
\noindent \textbf{Solution of the IP problem}\\
0
5
5 7.5
• • • • • •
• • • • •
• • • •
• • •
•
◦
❅
❅
❅
❅
❅
❅
❅
❅
❅
❅
❜❜❜❜❜❜❜❜❜❜❜❜❜❜❜
F1
F2













10 \end{frame}  
\begin{frame} 
\frametitle{Integer Programming}     
\noindent \textbf{Solution of the IP problem}\\
We can solve this graphically. The solution must lie at one of the
apexes of the feasible set.
The point of intersection of the two lines representing the linear
constraints is given by
x1 + x2=5
6x1 + 10x2=45.
This leads to x1 = 1.25, x2 = 3.75. The value obtained at this
point is 4 × 1.25 + 5 × 3.75 = 23.75.
11 \end{frame}  
\begin{frame} \frametitle{Integer Programming}     
Solution of the initial LP problem
The other apexes are (0, 0), (0, 4.5) and (5, 0) and the
corresponding values are 0, 22.5, 20.
It follows that the optimal solution of this LP problem is (1.25,
3.75) and the optimal value is 23.75. This value is an upper bound
on the optimal value in the integer programming (IP) problem.
In addition, since the coefficients of the objective function are
integers, then the value at any feasible point must also be an
integer. It immediately follows that 23 is an upper bound on the
optimal value in the IP problem.
12 \end{frame}  \begin{frame} \frametitle{Integer Programming}     
The initial branch
After the initial step, which produces an upper bound on the
optimal value, we carry out a branching step.
This involves restricting the values of one of the variables and
creating 2 new linear programming problems.
Suppose (x1, x2, . . . , xn) = (c1, c2, . . . , cn) is the optimal solution
of the present LP problem, where ci
is not an integer.
We add the constraint xi ≤ bci c to obtain one of the LP problems
and add the constraint xi ≥ dci e to obtain the other LP problem.
13 
\end{frame}  
\begin{frame} 
\frametitle{Integer Programming}     
\noindent \textbf{The two initial branch LPs}\\

In this case setting i = 1, the first LP problem obtained is thus
max z = 4x1 + 5x2
subject to
x1 + x2≤5
6x1 + 10x2≤45
x1≤1.
The feasible region of this problem is denoted in the diagram as F1.
% End of Slide 14 
\end{frame}  
\begin{frame} 
\frametitle{Integer Programming}     
\noindent \textbf{The two initial branch LPs}\\
The second LP problem is
max z = 4x1 + 5x2
subject to
x1 + x2≤5
6x1 + 10x2≤45
x1≥2.
The feasible region is denoted F2 in the diagram above.
The interior of the shaded area has been removed from our
considerations.
15 
\end{frame}  
\begin{frame} 
\frametitle{Integer Programming}     
\noindent \textbf{The Branching Mechanism}\\

This is the branching part of the algorithm. Each branch creates 2
LP problems from one LP problem and thus it may seem that the
number of LP problems considered increases. However,
\begin{itemize}
\item[1.] If the solution of one of the ”branch” LP problems is
a vector of integers, then we do not have to branch
any longer. Also, we can use bounds on the optimal
value to check whether a solution is or cannot be
optimal.
\item[2.] Many of the branches will lead to LP problems whose
set of feasible solutions is empty.
It can be shown that the branch and bound algorithm will always
come to a halt.
\end{itemize}
%% End of Slide 16 
\end{frame}  
%------------------------------------------------------%
\begin{frame} 
\frametitle{Integer Programming}     
Choice of way to split
It should be noted that after solving the initial LP problem, we can
form the next 2 LP problems by adding constraints on x2. In this
case, the first LP problem is
max z = 4x1 + 5x2
subject to
x1 + x2≤5
6x1 + 10x2≤45
x2≤3.
% End of Slide 17 
\end{frame} 
%================================================%
\begin{frame} 
\frametitle{Integer Programming}     
Choice of way to split
The second LP problem is given by
max z = 4x1 + 5x2
subject to
x1 + x2≤5
6x1 + 10x2≤45
x2≥4.
% End of Slide 18 
\end{frame}  
%==================================================%
\begin{frame} 
\frametitle{Integer Programming}     
Bounds on the optimal value and adding branches
1. If the solution of one of the ”branch” LP problems is
a vector of integers, then this gives a lower bound on
the optimal value. This branch does not need to
investigated any further.
2. If the solution of one of the ”branch” LP problems
contains a component that is not an integer (say
xi = c, where c ∈/ Z), then the optimal value of this
problem is an upper bound for the value of the IP
problem obtained in any the ”lower branch” problems
resulting from this one. Furthermore, if the
coefficients are integers and the variables must be
integers, the integer part of the value is an upper
bound on valid solutions found in ”lower branches”.
% End of Slide 19 
\end{frame}  
\begin{frame} 
\frametitle{Integer Programming}     
Bounds on the optimal value and adding branches
If we have already found a feasible solution whose value is greater
or equal to the upper bound on the value in a particular branch,
then this branch does not need to be investigated any further
(since splitting this branch will not lead to a better solution) .
Otherwise, we may split this branch into two LP problems. The
first is obtained by adding the constraint xi ≤ bcc, the second is
obtained by adding the constraint xi ≥ dce (where c is the value of
xi
in the optimal solution of the appropriate LP problem).
A branch is said to be presently active, if it has not yet been split
and a split is under consideration under the criterion above, or we
have not yet solved both LP problems of a branch that has been
split.
% End of Slide 20 
\end{frame}  
\begin{frame} 
\frametitle{Integer Programming}     
Bounds on the optimal value and adding branches
\begin{itemize}
\item[3.] The present upper bound on the objective function is
the maximum of the upper bounds found in the
branches that are presently active.
\item[4.] The present lower bound on the objective function is
the value of the best feasible solution found so far.
\end{itemize}
% End of Slide  21 
\end{frame}  
%=============================================%
\begin{frame} 
\frametitle{Integer Programming}     
\noindent \textbf{Example 1.1}\\\
Solve the IP problem.
max z = 4x1 + 5x2
subject to
x1 + x2 ≤ 5
6x1 + 10x2 ≤ 45,
where x1 and x2 are non-negative integers.
% End of Slide 22 
\end{frame}  
%===================================================%
\begin{frame} 
\frametitle{Integer Programming}     
Solution
The first part of the solution is to find an upper bound on the
objective function by solving the corresponding LP problem.
The solution to this problem has already been shown to be (1.25,
3.75) and the optimal value in this problem is 23.75. As argued
above, this means that 23 is an upper bound on the value in the IP
problem.
We now split this LP problem into 2 LP problems. The first is
obtained by adding the constraint x1 ≤ 1. The second is obtained
by adding the constraint x2 ≥ 2.
% End of Slide 23 
\end{frame}  
%===================================================%
\begin{frame} 
\frametitle{Integer Programming}     
\noindent \textbf{Solution}
We now consider these LP problems in turn. The apexes of the
feasible region F1 of the first problem (see diagram) are (0, 0), (1,
0), (0, 4.5) and (1, 3.9) [the intersection point of the lines x1 = 1
and 6x1 + 10x2 = 45].
The corresponding values of the objective function are 0, 4, 22.5
and 23.5. Hence, the optimal solution of this problem is (1, 3.9)
and the optimal value 23.5.
Arguing as above, since x2 is not an integer at this point, it follows
that 23 is an upper bound on the value of the IP problem obtained
in any of the branches starting from this problem.
% End of Slide 24 
\end{frame}  
%=========================================================%
\begin{frame}
\frametitle{Integer Programming}     
Solution
We now consider the second LP problem. The apexes of the
feasible region are (2, 0), (5, 0) and (2, 3). The corresponding
values of the objective function are 8, 20 and 23.
Hence, the optimal solution of this problem is (2, 3) and the
optimal value 23. Since, both components of the solution are
integers this is a lower bound on the optimal value in the IP
problem.
Since we have already shown that 23 is an upper bound on the
optimal value of the IP problem, it is clear that this solution is an
optimal solution to the IP problem.
% End of Slide 25 
\end{frame}  
%=========================================================%
\begin{frame}
\frametitle{Integer Programming}     
\noindent \textbf{Summary of the branch and bound procedure}
The branch and bound method may be summarised as follows
Figure: Summary of the branch and bound process for Example 1.1 26 \end{frame}  
\begin{frame} \frametitle{Integer Programming}     
\noindent \textbf{Note on choice of branches}
It is not clear what the optimal procedure for solving such a
problem should be. For example, if we first solved the LP problem
with feasible region F2, we would find the optimal solution
immediately.
However, in general after solving the LP problem with feasible
region F1, there is no reason why we cannot go on to split this
feasible region into 2.
Since the optimal solution to this LP problem is (1,3.9), the first
LP problem is obtained by adding the constraint x2 ≤ 3 and the
second LP problem is obtained by adding the constraint x2 ≥ 4.
% End of Slide 27 
\end{frame}  
%=========================================================%
\begin{frame} 
\frametitle{Integer Programming}     
\noindent \textbf{Note on choice of branches}
The apexes of the feasible region in the first problem are (0, 0), (1,
0), (0, 3) and (1, 3). The values corresponding to these points are
0, 4, 15 and 19. Hence, 19 is a lower bound on the optimal value.
The apexes of the feasible region in the second problem are (0, 4),
(0, 4.5) and ( 5
6
, 4). The values at these are 20, 22.5 and 23 1
3
.
This gives an upper bound of 23 for solutions to the IP problem in
this region. This region may then be split into two.
% End of Slide 28 
\end{frame}  
%=========================================================%
\begin{frame} 
\frametitle{Integer Programming}     
Note on choice of branches
It can be seen that the complexity of solution depends on the
choice of LP problem to be solved at each stage. There is no
general rule that guarantees that the optimal choice is made at
each stage.
One general rule of thumb would be to solve all the LP problems
of a given depth, starting with the branches associated with the
largest upper bounds (unless it becomes clear that the optimal
solution has been found), before splitting these LP problems.
The depth of an LP problem is understood to be the number of
splits required to obtain it from the original LP problem. Such a
procedure is likely to avoid the extremely long search times that
may result from exhausitively searching a non-optimal branch
before searching along other branches.
29 
\end{frame}  
%=========================================================%
\begin{frame}
\frametitle{Integer Programming}     
1.3 Mixed Integer/Linear Programming problems (IP/LP
problems)
In such problems some, but not all, of the variables are constrained
to be integers. The branch and bound algorithm can easily be
adapted to such problems.
Suppose (x1, x2, . . . , xk ) are constrained to be integers and
(xk+1, xk+2 . . . , xn) are not. It is assumed that the problem is a
maximisation problem.
30 \end{frame}  \begin{frame} \frametitle{Integer Programming}     
Mixed Integer/Linear Programming problems (IP/LP
problems)
At the first step of the algortithm, we solve the corresponding LP
problem in which none of the variables are constrained to be
integers.
The optimal value for this problem is an upper bound for the
optimal value in the IP/LP problem. If the values of (x1, x2, . . . xk )
are integers, then we have found the optimal solution.
Otherwise, we can split the initial LP problem into two problems as
before. If xi = c ∈/ Z, 1 ≤ i ≤ k at the optimal solution of this
problem, then the first LP problem is obtained by adding the
constraint xi ≤ bcc and the second LP problem is obtained by
adding the constraint xi ≥ dce.
%End of Slide 31 
\end{frame}  
%=============================================%
\begin{frame} 
\frametitle{Integer Programming}     
\noindent \textbf{The bounding procedure}
The bounding procedure is as for pure IP problems with the
following change:
1. If the components of the solution of one of the
”branch” LP problems that are supposed to be
integers are integers, then this gives a lower bound
on the optimal value. This branch does not need to
investigated any further.
32 
\end{frame}  
%=========================================================%
\begin{frame} 
\frametitle{Integer Programming}     
Example 1.2
Solve the mixed LP/IP problem.
max z = 4x1 + 5x2
subject to
x1 + x2 ≤ 5
6x1 + 10x2 ≤ 45,
where x1, x2 are non-negative and x2 is an integer.
%% End of Slide 33 
\end{frame}  
%=========================================================%
\begin{frame} 
\frametitle{Integer Programming}     
Solution
As before the first step is to solve the LP problem obtained by
ignoring the constraint that x2 is an integer. The solution of this
problem (as shown above) is (1.25, 3.75) and the optimal value is
23.75.
It should be noted that the optimal value in this problem is not
necessarily an integer and hence we cannot immediately improve
this upper bound as in the pure IP problem with integer
coefficients.
%% End of Slide 34 
\end{frame}  
\begin{frame} 
\frametitle{Integer Programming}     
Solution
We now split the initial linear programming problem into 2
problems. Since x1 does not have to be an integer, the two LP
problems are obtained as follows:
1. The first problem is obtained by adding the
constraint x2 ≤ 3 to the initial LP problem.
2. The second problem is obtained by adding the
constraint x2 ≥ 4 to the initial LP problem.
35 \end{frame}  \begin{frame} \frametitle{Integer Programming}     
Graphical solution
The problems obtained are illustrated below:
0
5
5 7.5
◦
❅
❅
❅
❅
❅
❅
❅
❅
❅
❅
❜❜❜❜❜❜❜❜❜❜❜❜❜❜❜
F1
F2
%% End of Sldie 36 
\end{frame}  
\begin{frame} 
\frametitle{Integer Programming}     
Solution
The apexes of F1 are (0,4), (0,4.5), ( 5
6
, 4) and the corresponding
values of the objective function are 20, 22.5, 23 1
3
.
Since x2 is an integer at the final solution, this gives us a lower
bound on the optimal value of the mixed IP/LP problem. We do
not have to split this LP problem into two.
The apexes of F2 are (0,0), (0,3), (2,3) and (5,0). The
corresponding values of the objective function are 0, 15, 23, 20. It
follows that the solution of this mixed IP/LP problem is ( 5
6
, 4) and
the optimal value 23 1
3
.
37 \end{frame}  \begin{frame} \frametitle{Integer Programming}     
1.4 The Travelling Salesperson Problem (TSP)
The problem is to find the shortest tour which visits all of n cities
and finishes in the starting city.
Without loss of generality, we can assume that the initial city is
labelled 1.
Suppose the distance between city i and j is di,j
. For convenience,
define di,i = ∞.
Let xi,j = 1 if a tour goes to city j after visiting city i, otherwise
xi,j = 0.
38 \end{frame}  
\begin{frame} 
\frametitle{Integer Programming}     
\noindent \textbf{The Travelling Salesperson Problem}\\
The mathematical formulation of the TSP is as follows: Minimize
z =
Xn
i=1
Xn
j=1
di,jxi,j
,
such that
Xn
j=1
xi,j = 1, i = 1, 2, . . . , n
Xn
i=1
xi,j = 1, j = 1, 2, . . . , n
xi,j ≥ 0 xi,j
is an integer
% End of Slid e39 
\end{frame}  
\begin{frame} 
\frametitle{Integer Programming}     
\noindent \textbf{The Travelling Salesperson Problem}\\

The first constraint states that exactly one city must be visited
immediately after city i.
The second constraint states that exactly one city must be visited
immediately before city j.
It follows that the TSP is an integer programming problem.
It should be noted that in the natural interpretation of this
problem di,j = dj,i
. However, the algorithms considered below can
be used to solved more general interpretations of the TSP.
The distances di,j are normally given in a matrix where di,j
is the
value in row i and column j.
%% End of Slide 40 
\end{frame}  
%=====================================================================================%
\begin{frame}
 \frametitle{Integer Programming}     

1.4.1 The Subtour Reversal Heuristic for Finding Good
Solutions of the TSP problem
The algorithm works as follows:
1. An initial tour is found by at each step travelling
from the present city to the nearest city that has not
yet been visited. When all the cities have been
visited, then we return to city 1.
2. We then see whether a shorter tour can be found by
reversing the order of two cities in the tour, then by
reversing the order of three cities (based on the
shortest tour found so far) and so on until the order
of n − 1 cities is reversed.
This algorithm will find a good solution, but not necessarily the
optimal solution.
41 
\end{frame}  
%=====================================================================================%
\begin{frame} 
\frametitle{Integer Programming}     
\noindent \textbf{Example 1.3}\\
Consider the following 5-city TSP problem given by the distance
matrix
D =


∞ 12 22 15 21
12 ∞ 8 11 13
22 8 ∞ 16 19
15 11 16 ∞ 19
21 13 19 19 ∞


42 \end{frame}  \begin{frame} \frametitle{Integer Programming}     
\noindent \textbf{Example 1.3}\\
First we find an initial tour. Starting from city 1, we go to city 2
(the closest). Distance 12
From city 2, we can go to any city except city 1. City 3 is the
closest. Distance 8
From city 3, we can go to city 4 or 5. City 4 is the closest.
Distance 16
From city 4, we must go to city 5. Distance 19.
Finally, we return to city 1. Distance 21.
%% End of Slide 43 
\end{frame}  
%===================================================%
\begin{frame} 
\frametitle{Integer Programming}     
\noindent\textbf{Example 1.3}\\
Hence, the initial tour is 1-2-3-4-5-1.
The length of this tour is 12+8+16+19+21=76.
We now consider reversing any two cities in the tour.
It should be noted that city 1 should not be moved, since the
resulting sequence does not define a tour.
%% End of Slide 44 
\end{frame}  
%====================================================%
\begin{frame} 
\frametitle{Integer Programming}     
\noindent \textbf{Example 1.3}\\
Hence, we consider the following reversals
Old Tour New Tour Length
1-2-3-4-5-1 1-3-2-4-5-1 22+8+11+19+21=81
1-2-3-4-5-1 1-2-4-3-5-1 12+11+16+19+21=79
1-2-3-4-5-1 1-2-3-5-4-1 12+8+19+19+21=79
None of these tours is better than the initial tour.
%% End of Slide 45 
\end{frame}  
\begin{frame} 
\frametitle{Integer Programming}     
\noindent \textbf{Example 1.3}\\
Next we consider the reversal of three cities in a tour in the best
tour found so far (the initial tour).
Old Tour New Tour Length
1-2-3-4-5-1 1-4-3-2-5-1 15+16+8+13+21=73
1-2-3-4-5-1 1-2-5-4-3-1 12+13+19+16+22=82
Hence, the best tour found so far is 1-4-3-2-5-1. Length 73.
46 \end{frame}  \begin{frame} \frametitle{Integer Programming}     
\noindent \textbf{Example 1.3}\\
Finally, we consider the reversal of four cities in this new tour.
Old Tour New Tour Length
1-4-3-2-5-1 1-5-2-3-4-1 21+13+8+16+15=73
Hence, we have found two tours (1-4-3-2-5-1 and 1-5-2-3-4-1)
which are of the same length.
It should be noted that if the distance matrix is symmetric, then
reversing n − 1 cities in a tour results in a tour of unchanged
length as the tour obtained is a pure reversal of the original tour.
47 \end{frame}  
\begin{frame} 
\frametitle{Integer Programming}     
1.4.2 The Branch and Bound Method for Solving the TSP
This method finds the optimal tour by solving an appropriate
sequence of assignment problems.
This approach uses the fact that every tour can be interpreted as
an assignment by assuming that individual i is assigned to task j
where j is the city which comes immediately after city i in the tour.
The cost of assigning individual i to task j is defined to be di,j
.
Since di,i = ∞, individual i will not be assigned to task i.
The tour 1-4-3-2-5-1 above can be interpreted as the following
assignment: Individual 1 does task 4, individual 4 does task 3,
individual 3 does task 2, individual 2 does task 5 and individual 5
does task 1.
48 
\end{frame}  
%=====================================================================================%
\begin{frame}
\frametitle{Integer Programming}     
1.4.2 The Branch and Bound Algorithm for Solving the
TSP
However, not every assignment can be interpreted as a tour. For
example, if individual 1 does task 3, individual 3 does task 4 and
individual 4 does task 1, then we have obtained a cycle which does
not visit all the cities (this assignment corresponds to the cycle
1-3-4-1).
In this case, individual 2 would be assigned to task 5 and individual
5 would be assigned to task 2 (corresponding to the cycle 2-5-2).
It can be shown that every assignment that cannot be interpreted
as a tour includes two or more cycles.
%% END OF SLIDE 49 
\end{frame}  
%=====================================================================================%
\begin{frame} 
\frametitle{Integer Programming}     
\noindent \textbf{
The Branch and Bound Algorithm for Solving the TSP}
The algorithm is as follows:
1. Since this is a cost minimisation problem, the value
of the solution to an assignment problem is a lower
bound on the optimal value of the TSP. A simple
upper bound is given by the length of the initial tour
using the subtour reversal heuristic.
2. Solve the assignment problem corresponding to the
distance matrix D. If the assignment corresponds to
a tour, then this is the optimal tour.
3. If the assignment does not correspond to a tour, then
it contains at least two cycles. We split the present
assignment problem into a set of new assignment
problems, each obtained by setting one of the
distances in a chosen (normally the shortest) cycle to
be ∞.
%% END OF SLIDE 50 
\end{frame}  
%=====================================================================================%
\begin{frame}
\frametitle{Integer Programming}     
The Branch and Bound Algorithm for Solving the TSP
4. These problems can be solved in the same way.
5. If the solution of an assignment problem does not
correspond to a tour, we can split it (if it seems
possible that a better solution than the best found so
far can be found) into a set of new assignment
problems as above.
6. If the solution of an assignment problem corresponds
to a tour, then we do not need to split this problem
and a we obtain an upper bound on the minimum
cost.
51 
\end{frame}  
%=====================================================================================%
\begin{frame}
\frametitle{Integer Programming}     
Example 1.4
Solve the TSP problem given by the matrix
D =


∞ 10 3 6 9
5 ∞ 5 4 2
4 9 ∞ 7 8
7 1 3 ∞ 4
3 2 6 5 ∞


52 \end{frame}  \begin{frame} \frametitle{Integer Programming}     
Example 1.4
We can find an initial upper bound on the minimum distance by
using the shortest distance heuristic.
From 1 we go to 3 (the nearest city). Distance 3. From 3 we go to
4. Distance 7.
From 4 we go to 2. Distance 1. From 2 we must go to 5. Distance
2.
From 5 we return to 1. Distance 3. The total distance is 16.
53 \end{frame}  \begin{frame} \frametitle{Integer Programming}     
Example 1.4
We now solve the corresponding assignment problem, call this A1.
First we subtract the minimum value in a row from the other row
values. This gives us the matrix


∞ 7 0 3 6
3 ∞ 3 2 0
0 5 ∞ 3 4
6 0 2 ∞ 3
1 0 4 3 ∞


54 \end{frame}  \begin{frame} \frametitle{Integer Programming}     
Example 1.4
Next we subtract the minimum value in a column from the other
column values. This gives us the matrix


∞ 7 0 1 6
3 ∞ 3 0 0
0 5 ∞ 1 4
6 0 2 ∞ 3
1 0 4 1 ∞


The sum of the numbers subtracted gives us a lower bound on the
cost of the optimal assignment (3+2+4+1+2+2=14).
55 \end{frame}  \begin{frame} \frametitle{Integer Programming}     
Example 1.4
It is not possible to choose 5 zeroes (5 is the number of cities) such
that there is one zero in each row and one zero in each column.
In this case, we highlight rows and columns such that all zeroes are
highlighted.
The minimum value of all the non-highlighted columns is
subtracted from the non-highlighted elements.
This value is added to the elements at the intersection of the
highlighted rows and columns.
56 \end{frame}  \begin{frame} \frametitle{Integer Programming}     
Example 1.4
Highlighting the appropriate rows and columns in red (those
doubly highlighted are in blue) leads to


∞ 7 0 1 6
3 ∞ 3 0 0
0 5 ∞ 1 4
6 0 2 ∞ 3
1 0 4 1 ∞


The minimum value of the non-highlighted (black) elements (1) is
subtracted from the black elements.
This element is added to the blue elements and the lower bound on
the optimal cost (which is now 14+1=15).
57 \end{frame}  \begin{frame} \frametitle{Integer Programming}     
Example 1.4
This gives


∞ 8 0 1 6
4 ∞ 3 0 0
0 5 ∞ 0 3
6 0 1 ∞ 2
1 0 3 0 ∞


There is a unique set of fives zeroes such that there is one zero in
each row and one in each column (marked in green).
The optimal assignment is 1 → 3, 2 → 5, 3 → 1, 4 → 2, 5 → 4.
The optimal cost is 15.
58 \end{frame}  \begin{frame} \frametitle{Integer Programming}     
Example 1.4
This assignment contains two cycles 1 → 3 → 1 and
2 → 5 → 4 → 2.
We obtain one new assignment problem, A2, by setting d1,3 = ∞
(the distance of one of the transitions in the shortest cycle).
The other assignment problem, A3, is obtained by setting
d3,1 = ∞.
59 \end{frame}  \begin{frame} \frametitle{Integer Programming}     
Example 1.4
The matrix corresponding to A2 is given by
D =


∞ 10 ∞ 6 9
5 ∞ 5 4 2
4 9 ∞ 7 8
7 1 3 ∞ 4
3 2 6 5 ∞


The solution to this assignment problem (the calculations are
analogous to those made above and are thus omitted) is 1 → 4,
2 → 5, 3 → 1, 4 → 3, 5 → 2 and the cost of this assignment is 17.
60 \end{frame}  \begin{frame} \frametitle{Integer Programming}     
Example 1.4
This assignment contains two cycles 1 → 4 → 3 → 1 and
2 → 5 → 2.
Hence, we could define two new assignment problems to solve.
One by setting d2,5 = ∞ and one by setting d5,2 = ∞.
However, the cost of this assignment is 17 (i.e. already more than
the solution found using the shortest distance heuristic). Hence,
we do not need to investigate this branch any more.
61 \end{frame}  \begin{frame} \frametitle{Integer Programming}     
Example 1.4
The matrix corresponding to A3 is given by
D =


∞ 10 3 6 9
5 ∞ 5 4 2
∞ 9 ∞ 7 8
7 1 3 ∞ 4
3 2 6 5 ∞


The solution to this assignment problem (details omitted) is 1 → 3,
2 → 5, 3 → 4, 4 → 2, 5 → 1 and the cost of this assignment is 16.
62 \end{frame}  \begin{frame} \frametitle{Integer Programming}     
Example 1.4
This solution corresponds to the tour 1 → 3 → 4 → 2 → 5 → 1
and so this branch does not need to be investigated any further.
Since it is impossible to find a better solution in the other branch,
we have found the optimal solution.
It should be noted that this is the solution found by the shortest
route heuristic. However, while the branch and bound algorithm
will always find an optimal solution, the shortest route and subtour
reversal heuristics are not guaranteed to do so.
63 \end{frame}  \begin{frame} \frametitle{Integer Programming}     
Example 1.4
The solution can be illustrated using a branch and bound tree as
shown below:
Figure: Summary of the branch and bound process for Example 1.4
64 \end{frame}  \begin{frame} \frametitle{Integer Programming}     
1.5 The Knapsack Problem
The problem is to load a rucksack in such a way as to maximize
the value of the objects in the rucksack while not exceeding the
maximum weight.
Suppose there are n types of object and ki objects of type i are
available.
The weight of objects of type i is wi
.
The value of objects of type i is vi
.
65 \end{frame}  \begin{frame} \frametitle{Integer Programming}     
The Knapsack Problem
Let xi be the number of objects of type i packed into the rucksack.
Let W be the maximum weight that can be carried.
In mathematical terms, we wish to maximise
z =
Xn
i=1
xi vi
subject to
Xn
i=1
wi ≤ W ; xi ∈ {0, 1, 2, . . . , ki}, 1 ≤ i ≤ n.
66 \end{frame}  \begin{frame} \frametitle{Integer Programming}     
The Knapsack Problem
It can thus be seen that the knapsack problem is an integer
programming problem.
The standard branch and bound technique can be used to find the
solution of this problem.
In this case, the form of the problem means that the associated
linear programming problems are easy to solve.
67 \end{frame}  \begin{frame} \frametitle{Integer Programming}     
The Knapsack Problem
Intuitively, if we could pack fractions of objects (i.e. turn the
problem into an LP problem), it would be optimal to start by
packing objects with the highest value to weight ratio and continue
until the rucksack is full.
If the solution found in this way involves packing an integer
number of objects of each type, then this is the optimal solution to
the IP problem.
68 \end{frame}  \begin{frame} \frametitle{Integer Programming}     
The Knapsack Problem
Otherwise, only a fraction of the final object is packed.
Suppose this object is of type i and xi = c. As in the general IP
problem, we can split the original LP problem into two different
problems. To obtain the first, we add the constraint xi ≤ bcc and
to obtain the second, we add the constraint xi ≥ dce.
These LP problems can be solved in the same way as the initial LP
problem.
Upper and lower bounds on the optimal value to the IP problem
are found in the same way as for general IP problems.
69 \end{frame}  \begin{frame} \frametitle{Integer Programming}     
Example 1.5
A lorry can carry 4 tonnes. Three types of object can be loaded.
Their values and weights (in tonnes) are given in the table below.
Item i vi wi
1 31 2
2 47 3
3 14 1
Find the load which maximises the value of the objects in the lorry.
70 \end{frame}  \begin{frame} \frametitle{Integer Programming}     
Example 1.5
In mathematical terms, the problem to be solved is to maximise
z = 31x1 + 47x2 + 14x3
subject to
2x1 + 3x2 + x3 ≤ 4.
Note that the maximum number of objects of type i that can be
packed is bW /wi c.
Hence, x1 ∈ {0, 1, 2}, x2 ∈ {0, 1} and x3 ∈ {0, 1, 2, 3, 4}.
71 \end{frame}  \begin{frame} \frametitle{Integer Programming}     
Example 1.5
We first solve the linear programming problem obtained by
allowing the xi to be non-negative reals.
In order to do this, we calculate the value to weight ratios for each
of the types of object. We have
r1 =
v1
w1
= 15.5; r2 =
v2
w2
= 15.667; r3 =
v3
w3
= 14
To find the optimal solution to the LP, we first load an object of
type 2 (the maximum number of such objects possible).
One tonne is still available, so we load half an object of type 1 (has
the 2nd highest value to weight ratio and weighs 2 tonnes).
72 \end{frame}  \begin{frame} \frametitle{Integer Programming}     
Example 1.5
Hence, the optimal solution to the initial LP problem is x1 = 0.5,
x2 = 1 and x3 = 0.
The value of this solution is 0.5 × 31 + 1 × 47 = 62.5.
This is an upper bound on the optimal value in the knapsack
problem.
Since the coefficients in the objective function are integers, the
optimal value must be an integer. Hence, this upper bound can be
immediately tightened to 62.
73 \end{frame}  \begin{frame} \frametitle{Integer Programming}     
Example 1.5
We now split this LP problem into two LP problems.
The first is obtained by adding the constraint x1 ≤ b0.5c. It follows
that given this constraint, we must have x1 = 0.
The second LP problem is obtained by adding the constraint
x1 ≥ d0.5e, i.e. x1 ≥ 1.
74 \end{frame}  \begin{frame} \frametitle{Integer Programming}     
Example 1.5
The first of these LP problems is to maximise
z = 47x2 + 14x3
subject to
3x2 + x3 ≤ 4.
This is solved as before. Since objects of type 2 have a higher
value to weight ratio, we first pack an object of type 2 (the
maximum number).
A load of one tonne is left, hence we can pack an object of type 3.
75 \end{frame}  \begin{frame} \frametitle{Integer Programming}     
Example 1.5
It follows that the optimal solution to this LP problem is
x1 = 0, x2 = 1, x3 = 1 and the value is 1 × 47 + 1 × 14 = 61.
Since the values of the decision variables are integers. This is a
lower bound on the optimal value.
76 \end{frame}  \begin{frame} \frametitle{Integer Programming}     
Example 1.5
We now solve the second LP problem. It should be noted that the
interpretation of the problem allows us to update the constraints of
the LP problem to be solved, which often allows us to find the
optimal solution more quickly.
In physical terms, the constraint x1 ≥ 1 means that we should first
load an object of type 2 (giving a reward of 31).
We then consider what other objects we can load.
The residual load available is 2 tonnes. It follows that no objects of
type 2 can be loaded, up to one object of type 1 and up to 2
objects of type 2.
77 \end{frame}  \begin{frame} \frametitle{Integer Programming}     
Example 1.5
This LP problem is thus to maximise
z = 31 + x
0
1 + 14x3
subject to
2x
0
1 + x3 ≤ 2,
where x
0
1
is the additional number of type 1 objects packed (above
1).
78 \end{frame}  \begin{frame} \frametitle{Integer Programming}     
Example 1.5
Since objects of type 1 have a higher value to weight ratio than
objects of type 3, we first load another type 1 object.
There is no residual load left. It follows that the optimal solution
to this problem is to load just 2 objects of type 1.
The value of this solution is 2 × 31 = 62.
Since this is a feasible solution which achieves the upper bound on
the objective function, this is the optimal solution.
79 \end{frame}  \begin{frame} \frametitle{Integer Programming}     
Example 1.5
The solution can be illustrated using a branch and bound tree as
shown below:
Figure: Summary of the branch and bound process for Example 1.5
80 \end{frame}  \begin{frame} \frametitle{Integer Programming}     
