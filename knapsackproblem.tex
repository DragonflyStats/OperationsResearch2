The knapsack problem or rucksack problem is a problem in combinatorial optimization: Given a set of items, each with a weight and a value, determine the number of each item to include in a collection so that the total weight is less than or equal to a given limit and the total value is as large as possible.

The knapsack problem or rucksack problem is a problem in combinatorial optimization: Given a set of items, each with a weight and a value, determine the number of each item to include in a collection so that the total weight is less than or equal to a given limit and the total value is as large as possible. It derives its name from the problem faced by someone who is constrained by a fixed-size knapsack and must fill it with the most valuable items.
The problem often arises in resource allocation where there are financial constraints and is studied in fields such as combinatorics, computer science, complexity theory, cryptography and applied mathematics.
The knapsack problem has been studied for more than a century, with early works dating as far back as 1897.[1] It is not known how the name "knapsack problem" originated, though the problem was referred to as such in the early works of mathematician Tobias Dantzig (1884–1956),[2] suggesting that the name could have existed in folklore before a mathematical problem had been fully defined.[3]

%==================================================================%

0/1 knapsack problem[edit]
A similar dynamic programming solution for the 0/1 knapsack problem also runs in pseudo-polynomial time. Assume w_1,\,w_2,\,\ldots,\,w_n,\, W are strictly positive integers. Define m[i,w] to be the maximum value that can be attained with weight less than or equal to w using items up to i (first i items).
We can define m[i,w] recursively as follows:
m[0,\,w]=0
m[i,\,w]=m[i-1,\,w] if w_i > w\,\! (the new item is more than the current weight limit)
m[i,\,w]=\max(m[i-1,\,w],\,m[i-1,w-w_i]+v_i) if w_i \leqslant w.
The solution can then be found by calculating m[n,W]. To do this efficiently we can use a table to store previous computations.

%=====================================================================%
One common variant is that each item can be chosen multiple times. The bounded knapsack problem specifies, for each item j, an upper bound uj (which may be a positive integer, or infinity) on the number of times item j can be selected:
maximize \sum_{j=1}^n p_j x_j	
subject to	\sum_{j=1}^n w_j x_j \leq W,	
 u_j \geq x_j \geq 0, x_j integral for all j
The unbounded knapsack problem (sometimes called the integer knapsack problem) does not put any upper bounds on the number of times an item may be selected:
maximize \sum_{j=1}^n p_j x_j	
subject to	\sum_{j=1}^n w_j x_j \leq W,	
 x_j \geq 0, x_j integral for all j
The unbounded variant was shown to be NP-complete in 1975 by Lueker.[1] Both the bounded and unbounded variants admit an FPTAS (essentially the same as the one used in the 0-1 knapsack problem).
