\documentclass[algocomplexity.tex]{subfiles}
\begin{document}

\section{Section 4 : Divide and Conquer approach}
%==================================%
\begin{frame}
	\frametitle{Divide and Conquer}
	\large
	\begin{itemize}
		\item n divide and conquer approach, the problem in hand, is divided into smaller sub-problems and then each problem is solved independently. 
		\item When we keep on dividing the sub-problems into even smaller sub-problems, we may eventually reach at a stage where no more dividation is possible. 
		\item Those "atomic" smallest possible sub-problem (fractions) are solved. The solution of all sub-problems is finally merged in order to obtain the solution of original problem.
	\end{itemize}
\end{frame}

%==================================%
\begin{frame}
	This algorithmic approach works recursively and conquer and merge steps works so close that they appear as one.
	
	
\end{frame}
%==================================%
\begin{frame}
	Examples
	The following computer algorithms are based on divide-and-conquer programming approach −
	Merge Sort
	Quick Sort
	Binary Search
	Strassen's Matrix Multiplication
	Closest pair (points)
	There are various ways available to solve any computer problem, but the mentioned are a good example of divide and conquer approach.
	
\end{frame}
%===================================================%
\end{document}