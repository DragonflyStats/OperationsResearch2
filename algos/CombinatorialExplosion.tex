\section{Combinatorial Explosions}
\begin{frame}
\frametitle{Combinatorial Explosions}
• Enumerating all possible solutions is of course not feasible
when there are too many items.
• What is “too many”?
– 500? 200? 100? 50? 10?
– Take a guess!
• Assume we can investigate 1 solution per cpu cycle at a
rate of 10 GHz (that’s 10 billion per second). Then,
enumerating all Knapsacks with 60 items takes more than
85 years!
\end{frame}
%=====================================================================%
• This effect is called a combinatorial explosion.
• If NP ≠ P, it cannot be avoided. However, we can aim at
pushing the intractable instance sizes as far as possible – far enough to solve real-world instances. This is what
\subsection{combinatorial optimization} is all about!

%=====================================================================%
