\documentclass[algocomplexity.tex]{subfiles}
\begin{document}
\section{Binary Search}
\begin{frame}
	%========================================================================%
	Data Structure - Binary Search
	%- http://www.tutorialspoint.com/data_structures_algorithms/binary_search_tree.htm
	
	Binary search is a fast search algorithm with run-time complexity of O(log n). This search algorithm works on the principle of divide and conquer. For this algorithm to work properly the data collection should be in sorted form.
\end{frame}

%==================================%
\begin{frame}
	\frametitle{Binary Search}
	\large
	Binary search search a particular item by comparing the middle most item of the collection. If match occurs then index of item is returned. If middle item is greater than item then item is searched in sub-array to the right of the middle item other wise item is search in sub-array to the left of the middle item. This process continues on sub-array as well until the size of subarray reduces to zero.
\end{frame}

%==================================%
\begin{frame}
	\frametitle{Binary Search}
	\large
	How binary search works?
	For a binary search to work, it is mandatory for the target array to be sorted. We shall learn the process of binary search with an pictorial example. The below given is our sorted array and assume that we need to search location of value 31 using binary search.
	
	Binary search
	First, we shall determine the half of the array by using this formula −
	
	mid = low + (high - low) / 2
	Here it is, 0 + (9 - 0 ) / 2 = 4 (integer value of 4.5). So 4 is the mid of array.
\end{frame}
%==================================%
\begin{frame}
	\frametitle{Binary Search}
	\large
	Binary search
	Now we compare the value stored at location 4, with the value being searched i.e. 31. We find that value at location 4 is 27, which is not a match. Because value is greater than 27 and we have a sorted array so we also know that target value must be in upper portion of the array.
\end{frame}
%==================================%
\begin{frame}
	\frametitle{Binary Search}
	\large
	Binary search
	We change our low to mid + 1 and find the new mid value again.
	
	low = mid + 1
	mid = low + (high - low) / 2
	Our new mid is 7 now. We compare the value stored at location 7 with our target value 31.
	
	Binary search
	The value stored at location 7 is not a match, rather it is less that what we are looking for. So the value must be in lower part from this location.
\end{frame}

%==================================%
\begin{frame}
	\frametitle{Binary Search}
	\large
	Binary search
	So we calculate the mid again. This time it is 5.
	
	Binary search
	We compare the value stored ad location 5 with our target value. We find that it is a match.
	
	Binary search
	We conclude that the target value 31 is stored at location 5.
	
	Binary search halves the searchable items and thus reduces the count of comparisons to be made to very less numbers.
\end{frame}
%==================================%
%\begin{frame}
%Pseudocode
%The pseudocode of binary search algorithm should look like this −
%
%Procedure binary_search
%   A ← sorted array
%   n ← size of array
%   x ← value ot be searched
%
%   Set lowerBound = 1
%   Set upperBound = n 
%
%   while x not found
%   
%      if upperBound < lowerBound 
%         EXIT: x does not exists.
%   
%      set midPoint = lowerBound + ( upperBound - lowerBound ) / 2
%      
%      if A[midPoint] < x
%         set lowerBound = midPoint + 1
%         
%      if A[midPoint] > x
%         set upperBound = midPoint - 1 
%
%      if A[midPoint] = x 
%         EXIT: x found at location midPoint
%
%   end while
%   
%end procedure
%\end{frame}
%==================================%
\begin{frame}
	
	Data Structure - Binary Search Tree
	
\end{frame}
%======================================%
\begin{frame}
	\frametitle{Binary Search}
	\large
	A binary search tree (BST) is a tree in which all nodes follows the below mentioned properties −
	
	\begin{itemize}
		\item The left sub-tree of a node has key less than or equal to its parent node's key.
		\item The right sub-tree of a node has key greater than or equal to its parent node's key.
	\end{itemize}
	Thus, a binary search tree (BST) divides all its sub-trees into two segments; left sub-tree and right sub-tree and can be defined as −
	
	\[left_subtree (keys)  \leq  node (key)  \leq  right_subtree (keys)\]
\end{frame}
%======================================%
\begin{frame}
	\frametitle{Representation of a Binary Search Tree}
	\large
	
	BST is a collection of nodes arranged in a way where they maintain BST properties. Each node has key and associated value. While searching, the desired key is compared to the keys in BST and if found, the associated value is retrieved.
	
	An example of BST −
\end{frame}
%======================================%
\begin{frame}
	Binary Search Tree
	We observe that the root node key (27) has all less-valued keys on the left sub-tree and higher valued keys on the right sub-tree.
\end{frame}
%======================================%
\begin{frame}
	Basic Operations
	Following are basic primary operations of a tree which are following.
	
	Search − search an element in a tree.
	
	Insert − insert an element in a tree.
	
	Preorder Traversal − traverse a tree in a preorder manner.
	
	Inorder Traversal − traverse a tree in an inorder manner.
	
	Postorder Traversal − traverse a tree in a postorder manner.
\end{frame}
%======================================%
%======================================%
\begin{frame}[fragile]
	Node
	Define a node having some data, references to its left and right child nodes.
	\begin{verbatim}
	struct node {
	int data;   
	struct node *leftChild;
	struct node *rightChild;
	};
	\end{verbatim}
\end{frame}

	
\end{document}

%=========================================================================%

\end{document}