Chapter 2: Dynamic Programming (DP)
%=======================================================================%

dynamic programming (also known as dynamic optimization) is a method for solving a complex problem by breaking it down into a collection of simpler subproblems, solving each of those subproblems just once, and storing their solutions

%=======================================================================%

2.1 Formulation of a DP Problem: The Optimality (Bellman)
Equations
Suppose you want to find the shortest route from Limerick to
Belfast.
First you could make a list of all the towns that you could pass
through and the distances between ”neighbouring” towns (it is
assumed that there is only one obvious path between such towns).
You could then calculate the lengths of all the possible routes and
choose the shortest one. However, this is not an efficient method.
1 \end{frame}  \begin{frame} \frametitle{Integer Programming}     
The Intuition of DP
The idea behind DP is to build up the optimal solution in stages.
This can be done inductively or recursively.
2 
\end{frame}  
\begin{frame} 
\frametitle{Integer Programming}     
Network (Graphical) Representation of the Problem
Suppose the possible routes can be described by a network. A
represents Limerick, K represents Belfast and the other letters
denote towns that can be visited along the route. The numbers on
each edge represent the distance between towns.







A
80
100
❅
❅
❅
❅
❅
❅
❅❅
90
B 70







C
60
90
❅
❅
❅
❅
❅
❅
❅❅
90
80D
90E
❅
❅
❅
❅
❅
❅
❅❅
80
70F
❅
❅
❅
❅
❅
❅
❅❅
60
80G
❅
❅
❅
❅
❅
❅
❅❅
90
H
80I







100
J
K
% End of Slide 3 
\end{frame}  
\begin{frame} 
\frametitle{Integer Programming}     
The Formulation of a DP Problem
In order to define a dynamic programming problem, we must define
i. The time horizon of the problem (number of decision
points/stages), denoted n.
Here this would be the maximum number of towns
that could be visited on the route between Limerick
and Belfast (4 e.g. A → B → E → H → K).
ii. Si
, the set of states that are possible at each stage,
i = 0, 1, 2, . . . , n.
Here Si
is the set of towns that can be the i-th
visited along a route. S0 = {s0}, where s0 is the
starting state and Sn is the set of possible finishing
states, i.e S1 ∈ {B, C, D} and S4 = K.
4 \end{frame}  \begin{frame} \frametitle{Integer Programming}     
Formulation-ctd.
iii) The set of possible actions at Stage i when in State
j, Ai,j
.
In this case, this is the set of paths that can be taken
from town j to neighbouring towns. The action of
moving from j to k will be denoted (j, k).
iv) The immediate costs (or benefits) of each possible
action.
Here, this cost is the distance cj,k between
neighbouring towns.
%End of Slide 5 
\end{frame}  
%=================================%
\begin{frame} 
\frametitle{Integer Programming}     
Formulation- ctd.
We have s0 = A, S1 = {B, C, D}, S2 = {E, F,G}, S3 = {H, I, J},
s4 = K (K is the only possible finishing state).
The problem can be described by a graph (network).
The states are nodes of the graph and the decision made at each
stage corresponds to an edge of the graph between two states
(here a path between two towns).
This correspondence is general for any deterministic problem.
%End of Slide 6 
\end{frame}  
%=========================================%
\begin{frame} 
\frametitle{Integer Programming}     
\noindent \textbf{2.1.1 The Inductive Procedure}
Working inductively, we would first calculate the shortest path
between Limerick and each of the towns that can be visited at
Stage 1 (this is banal).
When calculating the shortest subroute between Limerick and each
of the towns that can be visited at stage i + 1, we only need to
consider those routes which contain subroutes that are the shortest
from Limerick to the town visited at stage i.
%End of Slide 7 
\end{frame} 
%=======================================%
\begin{frame} 
\frametitle{Integer Programming}     
The Inductive Procedure
In mathematical terms, let di(j) be the the length of the shortest
subroute from Limerick to town j, which is visited at stage i, i.e.
j ∈ Si
.
Let k be a town that can be visited at stage i + 1 (i.e. k ∈ Si+1).
Let cj,k be the distance between ”neighbouring” towns j and k (if
j and k are not neighbours, then we may define cj,k = ∞).
In general, cj,k is the cost of the action which changes the state
from j to k in one stage.
8 \end{frame}  \begin{frame} \frametitle{Integer Programming}     
The Optimality/Bellman Equation
Since we are minimising costs, the optimality (Bellman) equation is
di+1(k) = min
j∈Si
{di(j) + cj,k }, (1)
where k ∈ Si+1.
When calculating the shortest subroute to k, we take the minimum
length route over the set of possible paths that come into k from
the previous town visited (the set of possible j), assuming that the
shortest possible subroute to j had been used).
9 \end{frame}  \begin{frame} \frametitle{Integer Programming}     
The Optimality/Bellman Equation
We should highlight the path coming into k which attains this
minimum distance. Once the calculation of the length of the
shortest route is complete. The optimal route can be traced
recursively as a continuous sequence of highlighted paths from the
finishing state to the initial state.
The minimum cost (length of the shortest route) is found by
minimising the costs of arriving at one of the finishing states
mins∈Sn dn(s). Here K is the only possible finishing state, thus the
minimum cost is simply dn(K).
10 \end{frame}  \begin{frame} \frametitle{Integer Programming}     
The Inductive Calculations
The appropriate calculations are as follows: These calculations can
be done on the graph. We should also highlight the paths for which
the minimum distance is achieved at each stage of the calculation.
At Stage 1, we travel to one of B, C and D. There is only one
path to each, so the lengths of the shortest routes at Stage 1 are:
d1(B) = 80, d1(C) = 100, d1(D) = 90. These three paths should
be highlighted.
At Stage 2, we can travel to E, F or G. There are 2 possible paths
to E, through B and through C. Hence,
d2(E)=min{d1(B) + cB,E , d1(C) + cC,E }
=min{80 + 70, 100 + 60} = 150. (2)
We highlight the path from B to E, since this minimises the
distance to E
11 \end{frame}  \begin{frame} \frametitle{Integer Programming}     
Inductive Calculation - ctd.
There is only 1 possible path to F, through C. Hence,
d2(F) = d1(C) + cC,F = 100 + 90 = 190.
We highlight the path from C to F.
There are 2 possible paths to G, through C and through D. Hence,
d2(G)=min{d1(C) + cC,G , d1(D) + cD,G }
=min{100 + 90, 90 + 80} = 170.
We highlight the path from D to G.
12 \end{frame}  \begin{frame} \frametitle{Integer Programming}     
Inductive Calculations- ctd.
At Stage 3, we can travel to H, I or J. There is only one path into
H, from E. Thus,
d3(H) = d2(E) + cE,H = 150 + 90 = 240.
We highlight the path from E to H.
There are 2 paths into I, from E and from F, we have
d3(I)=min{d2(E) + cE,I
, d2(F) + cF,I }
=min{150 + 80, 190 + 70} = 230. (3)
We highlight the path from E to I.
13 \end{frame}  \begin{frame} \frametitle{Integer Programming}     
Inductive Calculations- ctd.
There are 2 paths into J, from F and from G, we have
d3(J)=min{d2(F) + cF,J , d2(G) + cG,J }
=min{190 + 60, 170 + 80} = 250.
We highlight both these paths as the path lengths are equal.
14 \end{frame}  \begin{frame} \frametitle{Integer Programming}     
Inductive Calculations- ctd.
At the final stage we consider the three paths into Belfast, from
H, I and K. We have
d4(K)=min{d3(H) + cH,K , d3(I) + cI,K , d3(J) + cJ,K }
=min{240 + 90, 230 + 80, 250 + 100} = 310. (4)
We highlight the path from I to K.
15 \end{frame}  \begin{frame} \frametitle{Integer Programming}     
Derivation of the Optimal Solution
The optimal route (set of actions) can be found by starting at the
finishing state for which the minimum cost is obtained (here
Belfast), and recursively tracing the sequence of highlighted routes.
From Equation (4), the path to the final state minimising the total
distance used the path from I to K.
Working backwards, the path that minimised the distance to I
came from E [see Equation (3)]. The path that minimised the
distance to E came from B [see Equation (2)]. It follows that the
optimal route is A → B → E → I → K.
16 \end{frame}  \begin{frame} \frametitle{Integer Programming}     
Possibility of Multiple Solutions
There may not be a unique solution.
In such a case, at some stage of the recursive procedure defining
the optimal set of actions there will be two actions (paths) which
lead to the same cost (reward).
In this case either action (path) can be taken and any resulting
route defined by such a recursion will incur the same cost.
17 \end{frame}  \begin{frame} \frametitle{Integer Programming}     
Maximisation
Suppose instead of distances, the numbers represent the
attractiveness of a route.
In this case, let rj,k denote the attractiveness of a path between
towns j and k (if j and k are not neighbours, then we may define
rj,k = −∞).
The method is analogous. When calculating the most attractive
route between Limerick and the towns that can be visited at stage
i + 1, we only need to consider those paths which contain
subroutes that are the most attractive from Limerick to the town
visited at stage i.
18 \end{frame}  \begin{frame} \frametitle{Integer Programming}     
Bellman Equation for Maximisation
Let Ri(j) be the attractiveness of the most attractive subroute
between Limerick and town j, which is the i-th town visited.
It should be noted here that it is assumed that the attractiveness
of a route is the sum of the attractiveness of its component parts.
Calculating inductively, for k ∈ Si+1 the Bellman equation is
Ri+1(k) = max
j∈Si
{Ri(j) + rj,k }. (5)
19 \end{frame}  \begin{frame} \frametitle{Integer Programming}     
2.1.2 The Recursive Procedure
Obviously, the shortest route can be built up starting from Belfast
and working back to Limerick (i.e. recursively).
Let d
0
i
(j) be the distance of the shortest subroute from town j
(visited at Stage i) to Belfast.
Once we have calculated all the d
0
i
(j), we can then calculate all the
d
0
i−1
(k) for the towns that can be visited one step earlier on the
route.
20 \end{frame}  \begin{frame} \frametitle{Integer Programming}     
Bellman Equations for the Recursive Procedure
In this case, for k ∈ Si−1 the Bellman equations are of the form.
d
0
i−1
(k) = min
j∈Si
{d
0
i
(j) + cj,k }. (6)
i.e. we minimise the length of the subroute over the set of paths
coming out from k assuming that the optimal subroute is used
from the next stage onwards. The minimum cost is given by
d
0
0
(s0), where s0 is the initial state (here A).
21 \end{frame}  \begin{frame} \frametitle{Integer Programming}     
Bellman Equations for the Recursive Procedure- ctd.
In the case where we maximise the reward obtained, we have for
k ∈ Si−1
R
0
i−1
(k) = max
j∈Si
{R
0
i
(j) + rj,k }. (7)
When calculations are done recursively, then the optimal actions
are derived inductively (see the next example).
22 \end{frame}  \begin{frame} \frametitle{Integer Programming}     
Use of Dynamic Programming
It should be noted that dynamic programming is only appropriate
when the total costs (rewards) can be described as the sum of the
costs incurred (rewards gained) at each stage.
In certain cases the costs (rewards) can be transformed to satisfy
this condition. For example, if the total reward is the product of
the rewards gained at each stage, we wish to maximise Qn
i=1 ri
,
where ri
is the reward gained at the i-th stage.
This is equivalent to maximising
ln[Yn
i=1
ri
] = Xn
i=1
ln(ri).
23 \end{frame}  \begin{frame} \frametitle{Integer Programming}     
Use of Dynamic Programming-ctd.
Hence, by taking the logarithms of the component rewards, we can
define such a problem as a dynamic programming problem.
The dynamic programming procedure derives the logarithm of the
optimal reward.
24 \end{frame}  \begin{frame} \frametitle{Integer Programming}     
Example
Suppose the numbers given in the network from the previous
example give the probability (in percentage terms) of avoiding a
traffic jam along a route. Find the route which maximises the
probability of avoiding a traffic jam (assume that these
probabilities are mutually independent).
25 \end{frame}  \begin{frame} \frametitle{Integer Programming}     
Formulation
Taking the logarithms of the probabilities of being stuck in a traffic
jam gives the following network.







A
-0.22314
0
❅
❅
❅
❅
❅
❅
❅❅
-0.10536
B -0.35667







C
-0.51083
-0.10536
❅
❅
❅
❅
❅
❅
❅❅
-0.10536
-0.22314D
-0.10536E
❅
❅
❅
❅
❅
❅
❅❅
-0.22314
-0.35667F
❅
❅
❅
❅
❅
❅
❅❅
-0.51083
-0.22314G
❅
❅
❅
❅
❅
❅
❅❅
-0.10536
H
-0.22314I







0
J
K
26 \end{frame}  \begin{frame} \frametitle{Integer Programming}     
Formulation
The probability of not being stuck in a traffic jam is the product of
the probabilities along the chosen route.
In order to define the problem as a problem of dynamic
programming, the rewards gained at each stage are defined to be
the logarithms of the probabilities given above.
The optimal probability will be calculated by recursion. Let Ri(j)
be the logarithm of the maximum probability of not being caught
in a traffic jam when starting from Town j at Stage i. We have
Ri−1(k) = max
j∈Si
{Ri(j) + ln pj,k },
where pj,k is the probability of not being caught in a traffic jam
between Towns j and k.
27 \end{frame}  \begin{frame} \frametitle{Integer Programming}     
Calculation
We have R3(H) = ln(0.9) ≈ −0.10536,
R3(I) = ln(0.8) ≈ −0.22314, R3(J) = ln(1) = 0.
We highlight the three paths coming into K.
At Stage 2 we can be at E, F or G. From E we can go to either H
or I. Hence,
R2(E)=max{R3(H) + ln pE,H, R3(I) + ln pE,I }
≈max{−0.10536 − 0.10536, −0.22314 − 0.22314} = −0.21072.
We highlight the path from E to H.
28 \end{frame}  \begin{frame} \frametitle{Integer Programming}     
Calculation
From F we can go to I or J. Hence,
R2(F)=max{R3(I) + ln pF,I
, R3(J) + ln pF,J }
≈max{−0.22314 − 0.35667, 0 − 0.51083} = −0.51083.
We highlight the path from F to J.
From G we can only go to J. Hence,
R2(G) = R3(J) + ln pG,J ≈ −0.22314 + 0 = −0.22314.
We highlight the path from G to J.
29 \end{frame}  \begin{frame} \frametitle{Integer Programming}     
Calculation
At Stage 1, we can be at B, C or D. From B we can only go to E.
Hence,
R1(B) = R2(E) + ln pB,E ≈ −0.35667 − 0.21072 = −0.57982.
We highlight the path from B to E.
From C we can go to E, F or G. Hence,
R1(C)=max{R2(E) + ln pC,E , R2(F) + ln pC,F , R2(G) + ln pC,G }
≈max{−0.35667 − 0.51083, −0.51083 − 0.10536,
−0.22314 − 0.10536} = −0.32850 (8)
We highlight the path from C to G.
30 \end{frame}  \begin{frame} \frametitle{Integer Programming}     
Calculation-ctd.
From D we can only go to G. Hence,
R1(D) = R2(G) + ln pD,G ≈ −0.22314 − 0.22314 = −0.44629.
We highlight the path from D to G.
Finally, from A we go to one of B, C or D. Hence, we have
R0(A)=max{R1(B) + ln pA,B , R1(C) + ln pA,C , R1(D) + ln pA,D}
≈max{−0.57982 − 0.22314, −0.32850 + 0,
−0.44629 − 0.10536} = −0.32850. (9)
We highlight the path from A to C.
31 \end{frame}  \begin{frame} \frametitle{Integer Programming}     
Calculation-ctd.
This is the logarithm of the optimal probability of avoiding a traffic
jam.
Hence, the probability of avoiding a traffic jam is obtained by
carrying out the inverse operation (taking the exponent), i.e.
popt = e
−0.3285 = 0.72.
32 \end{frame}  \begin{frame} \frametitle{Integer Programming}     
Derivation of the Optimal Strategy
The optimal route can be derived by induction.
At the first stage, the maximum probability of avoiding a traffic
jam is attained when we travel to C [see Equation (9)].
From C the maximum probability is attained by travelling to G
[see Equation (8)]. From G we have to travel to J and then to K.
It follows that the optimal route is A → C → G → J → K. Of
course, the product of the probabilities along this path is the
optimal policy of avoiding a traffic jam.
33 \end{frame}  \begin{frame} \frametitle{Integer Programming}     
Formulation of Deterministic DP Problems as Linear
Programming (LP) Problems
The problem of finding the shortest route given above can be
defined as a linear programming problem as follows:
xj,k = 1 if the subroute between towns j and k is used and
xj,k = 0, otherwise.
The objective is thus to minimise P
j,k
cj,k xj,k , where cj,k is the
distance between j and k (this is the total distance of the route).
Other constraints are required to ensure that the route used is
valid, e.g. xA,B + xA,C + xA,D ≥ 1, i.e. one of the paths from
Limerick must be used.
34 \end{frame}  \begin{frame} \frametitle{Integer Programming}     
