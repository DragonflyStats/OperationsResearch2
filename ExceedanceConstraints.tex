\documentclass{beamer}

\usepackage{framed}

\begin{document}
%======================================%
\begin{frame}
	\noindent \textbf{Checklist}
	
	\begin{itemize}
		\item Picking Pivot Column
		\item Picking Pivot Row
		\item Picking Pivot Point
		\item Performing Elementary Row Operations
		\item Important: Recognize when iteration process is complete
		\item Recognize when optimal feasible solution has been found
		\item Recognize infeasibility 
	\end{itemize}
	
\end{frame}
%==============================================================%
\begin{frame}
	\noindent \textbf{Checklist}
	
	\begin{itemize}
		\item State the solution of Tableau(i.e. for LP relaxation)
		\item (Recognize which variables necessarily have a value of zero).
		\item Recognize that LP optimality does not equate to IP optimality.
	\end{itemize}
\end{frame}
%==============================================================%
\begin{frame}
	\noindent \textbf{Addition of Constraints to Simplex Tableau}
	
	\begin{itemize}
\item Important: Construction of New Constraints further to branch and bound.
\item This will involve adding new rows and columns to the tableau.
\item Remark: Exam 2012 Q1 Part D is very useful to practice with in this regard.
\item See next slide for exceedance constraints (i.e. $x_i \geq  k$)
\end{itemize}

\end{frame}
%==============================================================%
	\begin{frame}
		\frametitle{Branch and Bound : Encoding Exceedance Constraints}
		\large
	\noindent \textbf{Adding an Exceedence Constraint to a Tableau}
		\begin{itemize}
			\item Suppose we are to add a constraint such as $x_1 \geq 10$ to a simplex tableau
			\item We \textbf{subtract} an introduced slack variable (lets call it $x_5$).
			\item We can restate the constraint as \[x_1 -x_5 = 10\] (or equivalently $x_5-x_1=-10$)
			\item \textbf{\alert{BE CAREFUL WITH SIGNS!}}
		\end{itemize}
\end{frame}
%==============================================================%
\begin{frame}
	\frametitle{Branch and Bound : Encoding Exceedance Constraints}
	\large
Inserting this into an expanded simplex tableau
\begin{center}
	\begin{tabular}{|c||c|c|c|c|c|}
		
		\hline  $\ldots$ &$\ldots$  & $\ldots$ &$\ldots$  &$\ldots$  &$\ldots$  \\ 
	\hline  $\ldots$ &$\ldots$  & $\ldots$ &$\ldots$  &$\ldots$  &$\ldots$  \\ 
		\hline 10 & 1  & 0 & 0 & 0 & -1 \\ 
		\hline 
	\end{tabular} 
\end{center}
or equivalently
\begin{center}
	\begin{tabular}{|c||c|c|c|c|c|}
		\hline  $\ldots$ &$\ldots$  & $\ldots$ &$\ldots$  &$\ldots$  &$\ldots$  \\ 
		\hline  $\ldots$ &$\ldots$  & $\ldots$ &$\ldots$  &$\ldots$  &$\ldots$  \\ 
		\hline -10 & -1  & 0 & 0 & 0 & 1 \\
		\hline 
	\end{tabular} 
\end{center}
(Remark : Second version more useful as we are actually interested in slack variable specifically)
\end{frame}
\end{document}
