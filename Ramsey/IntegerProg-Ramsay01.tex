%%Operations Research 2
%%Lecturer: David Ramsey
%%Room B-2026
%%david.ramsey@ul.ie
%%www.ul.ie/ramsey

\documentclass{beamer}

\usepackage{amsmath}
%\usepackage(graphicx}
\usepackage{subfiles}
\begin{document}

%=========================================================%
\begin{frame} 
\frametitle{Integer Programming}     
Recommended Text
Operations Research- An Introduction. Hamdy A. Taha
(003/TAH)
The 7th edition is the most up to date in the library. The library
has 4 copies of this edition. In addition, there are around 25 copies
of older editions. Also,
Introduction to Operations Research. Hillier and Lieberman.
Course notes and tutorial sheets will be available on the Internet.
2 
\end{frame}  
%=========================================================%
\begin{frame} 
\frametitle{Integer Programming}     
Module Outline
\begin{description}
\item[1. Integer Programming.] Applications to the travelling
salesperson’s problem and the knapsack (packing) problem.
\item[2. Dynamic Programming.] Deterministic. Stochastic. Finite
horizon. Infinite horizon problems. Discounting and average-cost
problems.
\end{description}
\end{frame} 
%============================================================% 
\begin{frame} 
\frametitle{Integer Programming}     
CHAPTER 1 - INTEGER PROGRAMMING AND THE
BRANCH AND BOUND ALGORITHM
\begin{itemize}
\item 1.1 The definition of an integer programming problem
An integer programming problem can be defined by the description
of a linear programming problem together with the constraint that
the variables must be integers. It is assumed that the variables are
constrained to be non-negative.
\end{itemize}
% end of Slide 4 
\end{frame}  

%=====================================================================================%
\begin{frame}[fragile]
\frametitle{Integer Programming}     


Example of an integer programming problem
\begin{verbatim}
max z = 4x1 + 5x2
subject to
x1 + x2 ≤ 5
6x1 + 10x2 ≤ 45,
where x1, x2 are non-negative integers
\end{verbatim}
\begin{itemize}
	\item Note that we can describe any integer programming problem as a
	maximisation problem (the minimisation of 4x1 + 5x2 is equivalent
	to the maximisation of −4x1 − 5x2). 
	\item The function z = 4x1 + 5x2 is
	called the \textbf{\textit{objective function}}. 
\end{itemize}

% end of Slide 5 
\end{frame}  
%===================================================================%
\begin{frame} 
\frametitle{Integer Programming}     
\noindent \textbf{1.2 The Branch and Bound Method - The initial bound}
\begin{itemize}
\item 
\textbf{Important} We may find an initial upper bound on the optimal value in the
integer programming problem by solving the corresponding linear
programming problem.
\item Consider the problem given previously. If the values of x1 and x2 at the
optimal solution to this problem are integers, then this must be the
optimal solution to the integer programming problem.\\
\textit{(Remark: That is a big ``If".)}
\end{itemize}
% end of slide 6 
\end{frame}  
\begin{frame} 
\frametitle{Integer Programming}     
The Branch and Bound Method - The initial branching
step
\begin{itemize}
	\item Suppose x1 = c at this optimal solution, where c is not an integer.
	\item We can split the initial linear programming problem into two linear
	programming problems.
	\item The first problem is obtained by adding the constraint x1 ≤ bcc
	\textit{(bcc is the largest integer not greater than c)}.
	\item The second is obtained by adding the constraint x1 ≥ dce \textit{(dce is
	the smallest integer not less than c)}.
\end{itemize}
\end{frame}  
%===========================================================%
\begin{frame} 
\frametitle{Integer Programming}     
\noindent \textbf{The initial branching step}
\begin{itemize}
\item The union of the feasible sets for these two LP problems is the
feasible set for the initial LP problem with the set of (x1, x2)
satisfying $bcc < x1 < dce$ removed.
\item Since no feasible solutions in which both x1 and x2 are integers
have been removed (see diagram below, the dots show the feasible
points in the integer programming problem), the optimal solution
to the integer programming problem must lie in one of the feasible
regions of these two new problems.
\end{itemize} 
\end{frame}  

\end{document}
\begin{frame}[fragile] 
\frametitle{Integer Programming}     
\noindent \textbf{Solution of the IP problem}\\
Consider the linear programming problem
\begin{verbatim}
max z = 4x1 + 5x2
subject to
x1 + x2≤5
6x1 + 10x2≤45,
\end{verbatim}
The feasible region is illustrated in the figure on the next slide.

\end{frame}  
\begin{frame} 
\frametitle{Integer Programming}     
\noindent \textbf{Solution of the IP problem}\\
0
5
5 7.5
• • • • • •
• • • • •
• • • •
• • •
•
◦
❅
❅
❅
❅
❅
❅
❅
❅
❅
❅
❜❜❜❜❜❜❜❜❜❜❜❜❜❜❜
F1
F2
\end{frame}  
\begin{frame}[fragile]
\frametitle{Integer Programming}     
\noindent \textbf{Solution of the IP problem}\\
We can solve this graphically. The solution must lie at one of the
apexes of the feasible set.
The point of intersection of the two lines representing the linear
constraints is given by
\begin{verbatim}
x1 + x2=5
6x1 + 10x2=45.
This leads to x1 = 1.25, x2 = 3.75. The value obtained at this
point is 4 × 1.25 + 5 × 3.75 = 23.75.
\end{verbatim}
\end{frame}  
\begin{frame} \frametitle{Integer Programming}     
Solution of the initial LP problem
The other apexes are (0, 0), (0, 4.5) and (5, 0) and the
corresponding values are 0, 22.5, 20.
It follows that the optimal solution of this LP problem is (1.25,
3.75) and the optimal value is 23.75. This value is an upper bound
on the optimal value in the integer programming (IP) problem.
In addition, since the coefficients of the objective function are
integers, then the value at any feasible point must also be an
integer. It immediately follows that 23 is an upper bound on the
optimal value in the IP problem.
12 
\end{frame}  
\begin{frame} \frametitle{Integer Programming}     
The initial branch
\begin{itemize}
	\item After the initial step, which produces an upper bound on the
	optimal value, we carry out a branching step.
	\item This involves restricting the values of one of the variables and
	creating 2 new linear programming problems.
	\item Suppose (x1, x2, . . . , xn) = (c1, c2, . . . , cn) is the optimal solution
	of the present LP problem, where ci
	is not an integer.
	\item We add the constraint xi ≤ bci c to obtain one of the LP problems
	and add the constraint xi ≥ dci e to obtain the other LP problem.
\end{itemize}


\end{frame}  
\begin{frame} 
\frametitle{Integer Programming}     
\noindent \textbf{The two initial branch LPs}\\

In this case setting i = 1, the first LP problem obtained is thus
\begin{verbatim}
max z = 4x1 + 5x2
subject to
x1 + x2≤5
6x1 + 10x2≤45
x1≤1.
\end{verbatim}

The feasible region of this problem is denoted in the diagram as F1.
% End of Slide 14 
\end{frame}  
\begin{frame} 
\frametitle{Integer Programming}     
\noindent \textbf{The two initial branch LPs}\\
The second LP problem is
max z = 4x1 + 5x2
subject to
x1 + x2≤5
6x1 + 10x2≤45
x1≥2.
The feasible region is denoted F2 in the diagram above.
The interior of the shaded area has been removed from our
considerations.
15 
\end{frame}  
\begin{frame} 
\frametitle{Integer Programming}     
\noindent \textbf{The Branching Mechanism}\\

This is the branching part of the algorithm. Each branch creates 2
LP problems from one LP problem and thus it may seem that the
number of LP problems considered increases. However,
\begin{itemize}
\item[1.] If the solution of one of the ”branch” LP problems is
a vector of integers, then we do not have to branch
any longer. Also, we can use bounds on the optimal
value to check whether a solution is or cannot be
optimal.
\item[2.] Many of the branches will lead to LP problems whose
set of feasible solutions is empty.
It can be shown that the branch and bound algorithm will always
come to a halt.
\end{itemize}
%% End of Slide 16 
\end{frame}  
%------------------------------------------------------%
\begin{frame}[fragile]
\frametitle{Integer Programming}     
Choice of way to split
It should be noted that after solving the initial LP problem, we can
form the next 2 LP problems by adding constraints on x2. In this
case, the first LP problem is
max z = 4x1 + 5x2
subject to
x1 + x2≤5
6x1 + 10x2≤45
x2≤3.
% End of Slide 17 
\end{frame} 
%================================================%
\begin{frame}[fragile]
\frametitle{Integer Programming}     
Choice of way to split
The second LP problem is given by
\begin{verbatim}
max z = 4x1 + 5x2
subject to
x1 + x2≤5
6x1 + 10x2≤45
x2≥4.
\end{verbatim}

% End of Slide 18 
\end{frame}  
%==================================================%
\begin{frame} 
\frametitle{Integer Programming}     
Bounds on the optimal value and adding branches
\begin{enumerate}
	\item  If the solution of one of the ”branch” LP problems is
	a vector of integers, then this gives a lower bound on
	the optimal value. This branch does not need to
	investigated any further.
	\item If the solution of one of the ”branch” LP problems
	contains a component that is not an integer (say
	xi = c, where c ∈/ Z), then the optimal value of this
	problem is an upper bound for the value of the IP
	problem obtained in any the ”\textbf{lower branch}” problems
	resulting from this one.
\end{enumerate}
 
\end{frame}  
%==================================================%
\begin{frame} 
	\frametitle{Integer Programming} 
Furthermore, if the
coefficients are integers and the variables must be
integers, the integer part of the value is an upper
bound on valid solutions found in ”lower branches”.
% End of Slide 19 
\end{frame}  
\begin{frame} 
\frametitle{Integer Programming}     
\noindent \textbf{Bounds on the optimal value and adding branches}
\begin{itemize}
	\item If we have already found a feasible solution whose value is greater
	or equal to the upper bound on the value in a particular branch,
	then this branch does not need to be investigated any further
	(since splitting this branch will not lead to a better solution) .
	Otherwise, we may split this branch into two LP problems. 
	\item The
	first is obtained by adding the constraint xi ≤ bcc, the second is
	obtained by adding the constraint xi ≥ dce (where c is the value of
	xi
	in the optimal solution of the appropriate LP problem).
	\item A branch is said to be presently active, if it has not yet been split
	and a split is under consideration under the criterion above, or we
	have not yet solved both LP problems of a branch that has been
	split. 
\end{itemize}

% End of Slide 20 
\end{frame}  
\begin{frame} 
\frametitle{Integer Programming}     
Bounds on the optimal value and adding branches
\begin{itemize}
\item[3.] The present upper bound on the objective function is
the maximum of the upper bounds found in the
branches that are presently active.
\item[4.] The present lower bound on the objective function is
the value of the best feasible solution found so far.
\end{itemize}
% End of Slide  21 
\end{frame}  
%=============================================%
\begin{frame} [fragile]
\frametitle{Integer Programming}     
\noindent \textbf{Example 1.1}\\\
Solve the IP problem.
\begin{verbatim}
max z = 4x1 + 5x2
subject to
x1 + x2 ≤ 5
6x1 + 10x2 ≤ 45,
where x1 and x2 are non-negative integers.
\end{verbatim}
% End of Slide 22 
\end{frame}  
%===================================================%
\begin{frame} 
\frametitle{Integer Programming}     
Solution
The first part of the solution is to find an upper bound on the
objective function by solving the corresponding LP problem.
The solution to this problem has already been shown to be (1.25,
3.75) and the optimal value in this problem is 23.75. As argued
above, this means that 23 is an upper bound on the value in the IP
problem.
We now split this LP problem into 2 LP problems. The first is
obtained by adding the constraint x1 ≤ 1. The second is obtained
by adding the constraint x2 ≥ 2.
% End of Slide 23 
\end{frame}  
%===================================================%
\begin{frame} 
\frametitle{Integer Programming}     
\noindent \textbf{Solution}

We now consider these LP problems in turn. The apexes of the
feasible region F1 of the first problem (see diagram) are (0, 0), (1,
0), (0, 4.5) and (1, 3.9) [the intersection point of the lines x1 = 1
and 6x1 + 10x2 = 45].
The corresponding values of the objective function are 0, 4, 22.5
and 23.5. Hence, the optimal solution of this problem is (1, 3.9)
and the optimal value 23.5.
Arguing as above, since x2 is not an integer at this point, it follows
that 23 is an upper bound on the value of the IP problem obtained
in any of the branches starting from this problem.
% End of Slide 24 
\end{frame}  
%=========================================================%
\begin{frame}
\frametitle{Integer Programming}     
Solution
\begin{itemize}
	\item We now consider the second LP problem. The apexes of the
	feasible region are (2, 0), (5, 0) and (2, 3). The corresponding
	values of the objective function are 8, 20 and 23.
	\item Hence, the optimal solution of this problem is (2, 3) and the
	optimal value 23.
	\item  Since, both components of the solution are
	integers this is a lower bound on the optimal value in the IP
	problem.
	\item Since we have already shown that 23 is an upper bound on the
	optimal value of the IP problem, it is clear that this solution is an 	optimal solution to the IP problem.
\end{itemize}

% End of Slide 25 
\end{frame}  
%=========================================================%
\begin{frame}
\frametitle{Integer Programming}     
\noindent \textbf{Summary of the branch and bound procedure}
The branch and bound method may be summarised as follows
Figure: Summary of the branch and bound process for Example 1.1 26 \end{frame}  
\begin{frame} 
\frametitle{Integer Programming}     
\noindent \textbf{Note on choice of branches}
\begin{itemize}
	\item It is not clear what the optimal procedure for solving such a
	problem should be. For example, if we first solved the LP problem
	with feasible region F2, we would find the optimal solution
	immediately.
	\item However, in general after solving the LP problem with feasible
	region F1, there is no reason why we cannot go on to split this
	feasible region into 2.
	\item Since the optimal solution to this LP problem is (1,3.9), the first
	LP problem is obtained by adding the constraint $x2 \leq 3$ and the
	second LP problem is obtained by adding the constraint $x2 \geq 4$
\end{itemize}
.
% End of Slide 27 
\end{frame}  
%=========================================================%
\begin{frame} 
\frametitle{Integer Programming}     
\noindent \textbf{Note on choice of branches}
\begin{itemize}
	\item The apexes of the feasible region in the first problem are (0, 0), (1,
	0), (0, 3) and (1, 3). 
	\item The values corresponding to these points are
	0, 4, 15 and 19. 
	\item Hence, 19 is a lower bound on the optimal value.
	\item The apexes of the feasible region in the second problem are (0, 4),
	(0, 4.5) and ( 5
	6
	, 4). The values at these are 20, 22.5 and 23 1
	3
	.
	\item This gives an upper bound of 23 for solutions to the IP problem in
	this region. This region may then be split into two.
\end{itemize}

% End of Slide 28 
\end{frame}  
%=========================================================%
\begin{frame} 
\frametitle{Integer Programming}     
\noindent \textbf{Note on choice of branches}
\begin{itemize}
	\item It can be seen that the complexity of solution depends on the
	choice of LP problem to be solved at each stage. There is no
	general rule that guarantees that the optimal choice is made at
	each stage.
	\item One general rule of thumb would be to solve all the LP problems
	of a given depth, starting with the branches associated with the
	largest upper bounds (unless it becomes clear that the optimal
	solution has been found), before splitting these LP problems.
	\item The depth of an LP problem is understood to be the number of
	splits required to obtain it from the original LP problem. Such a
	procedure is likely to avoid the extremely long search times that
	may result from exhausitively searching a non-optimal branch
	before searching along other branches.
\end{itemize}
