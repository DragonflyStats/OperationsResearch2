

\documentclass[a4paper,12pt]{article}
%%%%%%%%%%%%%%%%%%%%%%%%%%%%%%%%%%%%%%%%%%%%%%%%%%%%%%%%%%%%%%%%%%%%%%%%%%%%%%%%%%%%%%%%%%%%%%%%%%%%%%%%%%%%%%%%%%%%%%%%%%%%%%%%%%%%%%%%%%%%%%%%%%%%%%%%%%%%%%%%%%%%%%%%%%%%%%%%%%%%%%%%%%%%%%%%%%%%%%%%%%%%%%%%%%%%%%%%%%%%%%%%%%%%%%%%%%%%%%%%%%%%%%%%%%%%
\usepackage{eurosym}
\usepackage{vmargin}
\usepackage{amsmath}
\usepackage{graphics}
\usepackage{epsfig}
\usepackage{subfigure}
\usepackage{fancyhdr}

%\usepackage{listings}
\usepackage{amsmath,enumerate,ifthen}
\usepackage{amssymb}
\usepackage{framed}
\usepackage{times}
%\usepackage{bigints}


\setcounter{MaxMatrixCols}{10}
%TCIDATA{OutputFilter=LATEX.DLL}
%TCIDATA{Version=5.00.0.2570}
%TCIDATA{<META NAME="SaveForMode" CONTENT="1">}
%TCIDATA{LastRevised=Wednesday, February 23, 2011 13:24:34}
%TCIDATA{<META NAME="GraphicsSave" CONTENT="32">}
%TCIDATA{Language=American English}

%\pagestyle{fancy}
%\setmarginsrb{20mm}{0mm}{20mm}{25mm}{12mm}{11mm}{0mm}{11mm}
%\lhead{MA4413} \rhead{Mr. Kevin O'Brien}
%\chead{Statistics For Computing}
%\input{tcilatex}

\begin{document}
\begin{center}
\includegraphics[scale=0.65]{shieldtransparent2}
\end{center}

\begin{center}
\vspace{1cm}
\large \bf {FACULTY OF SCIENCE AND ENGINEERING} \\[0.5cm]
\normalsize DEPARTMENT OF MATHEMATICS AND STATISTICS \\[1.25cm]
\large \bf {END OF SEMESTER EXAMINATION PAPER 2016} \\[1.5cm]
\end{center}

\begin{tabular}{ll}
MODULE CODE: MS4131 & SEMESTER: Spring 2016 \\[1cm]
MODULE TITLE: Linear Algebra 1 & DURATION OF EXAM: 2.5 hours \\[1cm]
LECTURER: Mr. Kevin O'Brien & GRADING SCHEME: 80 marks \\
& \phantom{GRADING SCHEME:} \footnotesize {100\% of module grade} \\[0.8cm]
EXTERNAL EXAMINER: Prof. J. King & \\
\end{tabular}
\bigskip
\begin{center}
{\bf INSTRUCTIONS TO CANDIDATES}
\end{center}

{\noindent \\ Scientific calculators approved by the University of Limerick can be used. \\
%Formula sheet and statistical tables are provided at the end of the exam paper.\\
Students must attempt any 4 questions from 5.}
%==========================================================================================%
\newpage
\large
\section*{Question 1}

	\subsection*{Part A. Matrix Multiplication (4 Marks)}	
	Given the matrices 
	$$
	A=\left(\begin{array}{cccc} 
	5&\!\!\!-1&4\end{array}
	\right), \qquad
	B =\left(\begin{array}{ccc} 
	2&1&7\\ \!\!\!-1&0&1\\0&1&3\end{array}
	\right), \qquad
	C=\left(\begin{array}{c} 3\\2\\ 5\end{array}
	\right),
	$$
	calculate the products $AB$ and $CA$.

	%----------------------------------------%
	\subsection*{Part B. Matrix Multiplication (5 Marks)}	
		Suppose A is a lower triangular matrix of the form;
		\[A = \left(
		\begin{matrix}
		a & 0 & 0 \\
		b & c & 0 \\
		d & e & f \\
		\end{matrix} \right)
		\]
		
		\begin{itemize}
			\item[(i)] (1 Mark) State the transpose of A.
			\item[(ii)] (3 Marks) Compute B where B = $ A \times A^{T}$. 
			\item[(iii)] (1 Mark) B is a symmetric matrix. What is meant by this?
		\end{itemize}
		\noindent 	\textit{Please Turn Over For Parts D and E}
			%----------------------------------------%
			\subsection*{Part C. Diagonal Matrices (3 Marks)}	
			Consider the following diagonal matrix D. Provide answers for the following questions in terms of the values $a$, $b$ and $c$.
			
			
			\[D = \left(\begin{array}{ccc}
			a & 0 & 0 \\ 
			0 & b & 0 \\ 
			0 & 0 & c
			\end{array} \right)\]
			\begin{itemize}
				\item[(i)] (1 Mark) Write an expression for the trace of the matrix D.
				\item[(ii)] (1 Mark) State the inverse of $D$, i.e. $D^{-1}$.
				\item[(iii)] (1 Mark) State the matrix $D^5$.
			\end{itemize}
			\smallskip
	\newpage
			\subsection*{Part D. Invertible Matrices (4 Marks)}	
	\noindent	Show that if $A$ is an $n\times n$ invertible matrix that satisfies 
	$$
	9A^3+A^2-3A=0
	$$
	where $A^n=\underbrace{A\ldots A}_{\textrm{$n$ times}}$, %_{\underbrace\textrm$n $ times$, 
	$I$ is the $n\times n$  identity matrix and $0$ is the $n\times n$  zero matrix,
	then the inverse of $A$ is given by  %\marks{4}
	$$
	A^{-1}=\frac13I+3A.
	$$
		\smallskip
		\subsection*{Part E. Matrix Transposition (4 Marks)}	
Let A and B be $m \times n$ matrices. 
	
\[	(AB)^{T} = B^T \times A^T\]

\begin{itemize}
\item[(i)] (4 Marks) Prove this identity for $A$ and $B$.\\ \smallskip \textit{A proof that is provided on the basis that $A$ and $B$ are both $2 \times 2$ matrices will be sufficient for full marks.}
\end{itemize}
%==========================================================================================%
\newpage
\section*{Question 2}
	%===========================================================================%
	\subsection*{Part A. Fundamental Theorem of Invertible Matrices  (4 Marks)}
	The Fundamental Theorem of Invertible Matrices states that a set of mathematical expressions concerning an $n\times n$ matrix $A$ are each equivalent to one another.
	
	\begin{itemize}
		\item[(i)] ($4 \times 1$ Mark)
		State any four of these expressions.
		%\item[(i)] (1 Mark) What is the trace of a square matrix
%		\item[(ii)] (1 Mark) What is the rank of a matrix.
	\end{itemize}
	
	\subsection*{Part B. Inverting a Matrix with Elementary Row Operations  (6 Marks)}	
In this question, you are required to find the inverse of the following matrix using elementary row operations.
	
	\begin{equation*}
	A=\left( \begin{array}{rrr}
  1 &  -4   & -3 \\
  1 &   3   & 5 \\
  -2 &   0   & -4 \\
	\end{array} \right)
	\end{equation*}
	
	
			\begin{itemize}
				\item[(i)] (2 Mark) Write down the augmented matrix of this system. %\marks{4}
				
				%			\item[(ii)] (1 Mark) What can you say about the solution set of the system? Justify your answer. %\marks{4}
				
				\item[(ii)] (4 Marks) Find the inverse of the matrix, using elementary row operations. Show your workings for each stage of the calculation.
			\end{itemize}
			\medskip
		\noindent 	\textit{Please Turn Over For Parts C and D}
\newpage
	%---------------------------------------------------%
	\subsection*{Part C. Inverting a Matrix with Co-Factor Method (9 Marks)}		
		\begin{equation*}
		B=\left( \begin{array}{rrr}
4  &  3 &  -1 \\
-5 &  -3  &  1 \\
-2 &   3 &  -2 \\
		\end{array} \right)
		\end{equation*}
	\begin{itemize}
		\item[(i)] (5 Marks) For each element of $B$, calculate the corresponding minor. Show your workings for each calculation. 
		State the matrix of minors.
		\item[(ii)] (2 Marks) Hence or otherwise, compute the determinant of $B$ i.e. $\det(B)$.
		\item[(iii)] (1 Mark) Compute the cofactor matrix for $B$ i.e. $\operatorname{cof}(B)$.
		\item[(iv)] (1 Marks) State the inverse matrix of $B$, given by
		\[ B^{-1}=\frac{1}{\det(B)}  \operatorname{cof}(B)^T. \]
	\end{itemize}
	\subsection*{Part D. Inverting Multiples of a Matrix  (1 Mark)}	
	Suppose that the inverse of the following matrix $M$ 
	\[M = \left(\begin{array}{rrr}
	2    & 2  &  2 \\
	4    & 0 &  -2\\
	-6   & -2 &   2\\
	\end{array}\right)\]
	is given as 
		\[M^{-1} = \left(\begin{array}{rrr}
0.25 & 0.5  & 0.25\\
-0.25 & -1.0& -0.75\\
 0.50 & 0.5 & 0.50\\
		\end{array}\right)\]
		
			\begin{itemize}
				\item[(i)] (1 Mark) State the inverse of the matrix $N$ where $N = 2M$.
				\end{itemize}
	\[
	N = 2M = \left(\begin{array}{rrr}
	4 &   4  &  4 \\
	8 &   0  & -4\\
	-12 &  -4  &  4\\
	\end{array}\right)
	\]
%==========================================================================================%
\newpage
\section*{Question 3}
\subsection*{Part A. Vector Calculations (7 Marks)}
Consider the three vectors in $\mathbb{R}^3$:
$$
u = (1, 2, 4), \quad v = (0, 4, 3),\quad w = (-4, 2, 3).
$$
\begin{itemize}
	\item[(i)] (2 Marks) Evaluate $\|u\|$ and $\|v\|$,
	\smallskip \item[(ii)] (3 Marks) Evaluate $u\cdot v$, $u\times v$ and the angle between $u$ and $v$. %\marks{3}
	
	\smallskip\item[(iii)] (2 Marks) Calculate the scalar triple product  $u\cdot(v \times w)$.%\marks{3}

\end{itemize}
\smallskip
\subsection*{Part B. Orthonormal Projections (7 Marks)}
%------------------------------------------------%
%	
%	Orthonormal Projection
%	Orthogonal Projection
%================================================ %
If
\begin{equation*}
u =\left[ \begin{array}{c} 1 \\ -1 \\ 3 \end{array}\right],\qquad 
a =\left[ \begin{array}{c} 3 \\ 4 \\ 7 \end{array}\right],
\end{equation*}
%	\[u=(1,0,3);\qquad a=(3,5,7,)\]

\begin{itemize}
	\item[(i)](3 Marks) Find the vector component of $u$ along $a$, $\operatorname{proj}_{a}u$ 
	\item[(ii)](2 Marks) Find the
	vector orthogonal component to $a$;
	
	\item[(iii)](2 Marks) Calculate the norm of $\operatorname{proj}_a u$ and the norm of $u-\operatorname{proj}_a u$;
	
%	\item[(iii)]   Draw $proj_a u$ showing its direction and orientation.
\end{itemize}
%\subsection*{Part C. Vector Proofs (5 Marks) }		
%Prove that, for any $u,\:v\:\in\mathbb{R}^3$, %\marks{8}
%$$(u\times v)\times w= (u\cdot w)v - (v\cdot w)u.$$
%(It is sufficient to verify this property for one component.)
	\smallskip
	\subsection*{Part C. Proofs for Vector Products (6 Marks)}
	
	\begin{itemize}
		
		\item[(i)] (3 Marks) Prove that, for any $u,\:v\:\in\mathbb{R}^3$\[(u+v)\times w=u\times w + v\times w.\]
		
			\item[(ii)] (3 Marks) Prove that, for any $u,\:v\:\in\mathbb{R}^3$
			
			\[(u\times v)=(ku)\times v = u\times (kv).\]
	\end{itemize} \vspace{0.4cm}

%==========================================================================================%
\newpage
 \section*{Question 4}
	%=================================================================%
	\subsection*{Part A. System of Linear Equations (7 Marks)}
 Consider the linear system
		\begin{align*}
		x_1 + x_3 &= 4\\
		2x_1 + 4x_2 + x_3 &= -3\\
		x_2 + 3x_3 &= 7.
		\end{align*}
		\begin{itemize}
			\item[(i)] (2 Marks) Write down the coefficient matrix and the augmented matrix of this system.  %\marks{4}
			
%			\item[(ii)] (1 Mark) What can you say about the solution set of the system? Justify your answer. %\marks{4}
			
			\item[(ii)] (5 Marks) Solve the system of equations, using any appropriate method. Show your workings for each stage of the calculation.
		\end{itemize}
%		\item Consider the homogeneous system:
%		\begin{align*}
%		x_1 + x_3 &= 0\\
%		2x_1 + 4x_2 + x_3 &= 0\\
%		x_2 + 3x_3 &= 0.
%		\end{align*}
%		What can you say about its solution set?%\marks{4}



	\subsection*{Part B. Proof of Vector Identities (7 Marks)}
	% MS4131 Autumn 2009 / 2010
	% Question 2b
	\begin{itemize}
		\item[(i)] (1 Marks) Let u and v be two vectors in $\mathbb{R}^3$ and let $\theta$ be the angle between them. Define the scalar product $u \cdot v$ in terms of $\theta$ 
		\item[(i)] (2 Marks) Hence or otherwise, prove the so-called ``\textit{Cauchy-Schwarz Inequality}":
		
		\[ \|u \cdot v \|  \leq \|u \|\times \| v \|  \]
		\item[(ii)] (4 Marks) Hence or otherwise, prove the so-called ``\textit{Triangular Inequality}"
		\[ \|u + v\|  \leq  \|u \| +  \| v \| \]
		
%		\item[(iii)] Let $v= (v_1, v_2,v_3)$ be a vector in $\mathbb{R}^3$ and $k$ be a scale
%		( $k \in \mathbb(R)$ and $k \geq 0$). Show that
%		
%		\[ \| kv \| =  k \|v\|\], where $\|v\|$ denotes the euclidena norm.
	\end{itemize}
	
	%%=====================================%
	%\subsection*{Part C. Addition of Vectors}
	%%Tutorial sheet?
	%Let u = (-1, 2, 3, 1) and v = (1, 0, 5, -2) be two vectors in $\mathbb{R}^4$. 
	%% Evaluate $\|u\|$, $\|v\|$, $\|u + v\|$ and . 
	%\begin{itemize}
	%	\item[(i)] Check that u and v satisfy the triangle inequality. 
	%\end{itemize}
	


\subsection*{Part C. Distance from Planes (6 Marks)}	

%2011 Question 2
\begin{itemize}
	\item[(i)] (3 Marks) Give the general form of the equation of the plant $\pi$ in $\mathbb{R}^3$ passing through the point $P_0 =(1,0,2)$ with the vector $n=(-5,5,2)$ as the normal.
	
	
	\item[(ii)] (3 Marks) Show that the point $Q=(1,-1,1)$ does not lie in the plane $\pi$ and find its distance from $\pi$.
\end{itemize}

%\subsection*{Part D. Planes (5	Marks)}
%\begin{enumerate}
%	\item Find the general form of the equation of the plane $\pi$ in $\mathbb{R}^3$ which passes through the point 
%	$P=(3,1,6)$ and is orthogonal to the vector $n=(1,7,-2)$. %\marks{3}
%	
%	\item Show that the point $Q=(1,-1,1)$ does not lie in the plane $\pi$ and find its distance from $\pi$. %\marks{ 2}
%\end{enumerate}
%=========================================================%
\section*{Question 5}

	\subsection*{Part A. Eigenvalues and Eigenvectos (12 Marks)}
	Consider the following matrix $A$ 
	\[
	A = \left(\begin{array}{rrr} 
	0.5 & 1 & 1.5 \\
	-3  & -5 & -3\\
	3.5 & 5  & 2.5 \\ \end{array}\right)
	\]	
	\begin{itemize}
		\item[(i)] (2 Marks) Determine the Characteristic Equation for $A$.
		\item[(ii)] (2 Marks) Determine the eigenvalues for $A$.
		\item[(iii)] (4 Marks) For each eigenvalues, compute the corresponding eigenvectors of $A$.
		
		\medskip\item[(iv)] (2 Marks) Diagonalise $A$; i.e, give a matrix $P$ and a diagonal matrix $D$, such that $A=PDP^{-1}$.
		
		\medskip \item[(v)] (2 Marks) Hence, evaluate $A^4$.
	\end{itemize} \vspace{0.4cm}
	
	
	
	
	%========================================%
\subsection*{Part B. Determinants of Matrices (4 Marks)}

You are given the following piece of information concerning a matrix $A$.
\[|A| = \begin{vmatrix}
0  &  4   & 7 \\
-1  & -1  &  7\\
1   & 5  &  1\\
\end{vmatrix} = 4 \]

\begin{itemize}
	\item[(i)] (2 Marks) Hence or otherwise, state the determinant of matrices $B$ and $C$. Provide a brief justification for your answer.
	\[B = \left(\begin{array}{rrr}
	0  &  4   & 7 \\
	-3  & -1  &  7\\
	3   & 5  &  1\\
	\end{array}\right) \qquad C = \left(\begin{array}{rrr}
	0  &  8   & 14 \\
	-2  & -2  &  14\\
	2   & 10  &  2\\
	\end{array}\right) \]
	
	\item[(ii)] (2 Marks) State the determinant of matrices $D$ and $E$. Provide a brief justification for your answer.
	\[D = \left(\begin{array}{rrr}
	0  &  4   & 8 \\
	-3  & -1  &  -2\\
	3   & 5  &  10\\
	\end{array}\right) \qquad E = \left(\begin{array}{rrr}
	0  &  8   & 14 \\
	0  & -2  &  14\\
	0   & 10  &  2\\
	\end{array}\right) \]
\end{itemize}


%==========================================================================================%




\subsection*{Part C. Row-Echelon Form of a Matrix (4 Marks)}
Consider the matrices $U,V,W$ and $X$ presented below. For each matrix state one reason why that matrix in not in row-echelon form. Provide distinct answers for each of the four matrices.
\[
U = \begin{pmatrix}
1&2 & 6  &3 & 5
\\  0&1&4 &0 & 6
\\  0&1&2 &1 & 7
\\  0&0&1 &1 & 7
\end{pmatrix} \qquad  V = \begin{pmatrix}
1&1 & 6  &8 & 5
\\  0&2&6 &0 & 6
\\  0&0&1 &1 & 7
\\  0&0&0 &1 & 7
\end{pmatrix}
\]

\[
W = \begin{pmatrix}
1&1 & 6  &8 & 5
\\  0&1&6 &0 & 6
\\  0&0&0 &1 & 7
\\  0&0&1 &1 & 7
\end{pmatrix} \qquad X = \begin{pmatrix}
0&0&0 &0 & 0\\
1&1 & 6  &8 & 5
\\  0&1&6 &0 & 6
\\  0&0&0 &0 & 1

\end{pmatrix}
\]

\smallskip
\noindent \textbf{Marking Scheme}: \textit{$4 \times 1$ Marks where 1 Mark is awarded for each valid and distinct reason}.\\
\medskip

\noindent 	\textit{Please Turn Over For Part C }

%=================================================================%


\end{document}

%============================================%
A = matrix(sample(-5:7,9,replace=T),nrow=3)
det(A)

while(det(A) != 3){
	Atop =c(1,1,1)
	Aelements=c(Atop,sample(-5:7,6,replace=T) )
    A = matrix(Aelements),byrow=T, nrow=3)
   }







